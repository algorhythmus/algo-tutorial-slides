%beamer

%\PassOptionsToClass{handout}{beamer}

\input{../preamble/tutpreamble}

\newcommand{\Knapsack}{\textsc{Knapsack}\xspace}

\begin{document}
	
\starttut{12}
	
\begin{frame}{Schwarzes Brett + Klausurinfos!}
	\begin{itemize}
		\item {\Large \textbf{\{ \}} vs. \textbf{( )}} – Ungerichtete/gerichtete Kanten
		\item \textbf{Klausur} findet statt am \textbf{04.09.2017} von \textbf{11–13~Uhr} \\
		\item \textbf{Erlaubt}: Stifte, 4-Gänge-Menü, \textbf{Cheatsheet} (1 DIN-A4-Blatt beidseitig beliebig beschrieben) 
		\item Klausur\textbf{anmeldung} bis 28.08.17. \\
		Klausur\textbf{abmeldung} bis 28.08.17, danach nur \textbf{direkt} vor Klausur im HS!
	\end{itemize}
\end{frame}
	
\begin{headframe}[First World Problems]
	Optimierungsprobleme
\end{headframe}
	
\begin{frame}{Optimierungsprobleme}
	\textbf{Mehr Effizienz} 
	\begin{itemize}
		\item Dijkstra: \textbf{Kürzeste} Pfade
		\item Jarník-Prim bzw. Kruskal: \textbf{Minimale} Spannbäume
		\item[...]
		\implitem Alles \textbf{Optimierungsprobleme}
		\pause
		\item \textbf{Heute}: Optimierungsprobleme \textbf{allgemein} – und wie man sie \textbf{löst} \\
	\end{itemize}
\end{frame}

\begin{frame}{Optimierungsprobleme – \Knapsack}
	\textbf{Beispiel: Ich nehme meinen Rucksack und packe ein...} 
	\begin{itemize}
		\item \textbf{Rucksackproblem} (\Knapsack): \\
		\textbf{Gegeben}: \\ 
		\quad Rucksackplatz $M$, \\
		\quad $n$ Gegenstände mit \textbf{Gewicht} $w_i$ und \textbf{Profit} $p_i$ \\
		\textbf{Gesucht}: Teilmenge $X$ der Gegenstände, sodass \\
		\quad $\sum\limits_{i \in X} p_i$ \textbf{maximal} wird, aber $\sum\limits_{i \in X} w_i \leq M$ bleibt
		\item \textbf{Nicht alle} Gegenstände passen in den Rucksack
	\end{itemize}
	\forcenewline
	\pause
	Lösungsansätze?
\end{frame}

\begin{frame}{Optimierung – Greedy}
	\textbf{Wir bereuen nichts: Greedy-Algorithmen} 
	\begin{itemize}
		\item Prinzip: Reine \textbf{Gier}, never step back! \\
		Was \textbf{grad} am \textbf{Besten} scheint: \textbf{Direkt} nehmen!
		\pause
		\implitem Kann in \textbf{Sackgasse} führen 
		\implitem Auf die \textbf{Spitze} geht's manchmal nur durchs \textbf{Tal}
		\pause
		\item Kann aber auch funktionieren: \\
		Dijkstra, Jarník-Prim, Kruskal – alles \textbf{greedy} und läuft \yop
	\end{itemize}
	\pause
	
	Ein \textbf{Greedy-Algo} für \textbf{\Knapsack}:
	\begin{itemize}
		\item Schmeiße der Reihe nach Gegenstände mit \textbf{bestem} Profit-/Gewicht-\textbf{Verhältnis} $\frac{p_i}{w_i}$ rein, bis voll
		\pause
		\implitem Aber: \textbf{nicht optimal}, Bsp.: \ $M=10, \ (p_i, w_i) = (8, 6), (5, 5), (5, 5)$ \crash
		\implitem Greedy-Algorithmus für \Knapsack \textbf{ungeeignet}, kann sich eine optimale Lösung \textbf{verbauen}\\
	\end{itemize}
\end{frame}

\begin{frame}{Optimierung – DP}
	\textbf{Rekursion rückwärts: Dynamic Programming (DP)} 
	\begin{itemize}
		\item \textbf{Teile-und-Herrsche}: \\
		Löse großes Problem durch \textbf{Zerlegung} in Kleinere
		\pause
		\item \textbf{Dynamic Programming}: \\
		Löse \textbf{Kleinere} zuerst, setze dann zu \textbf{Größeren} zusammen: \\
		\pause
		\quad Konstruiere optimale „\textbf{Minimallösungen}“ \~~> zu größeren \\ 
		\quad Optimallösungen \textbf{erweitern} \~~> bis zum urspr. Problem.
		\pause
		\item Meistens \textbf{zweidimensional}: Tabelle mit \textbf{Rekursionsformel} ausfüllen. (siehe Beispiel)
		\pause
		\item Formal: DP ist anwendbar \gdw die optimale Lösung besteht aus optimalen Lösungen von \textbf{Teilproblemen}.
		%\item D.h. konstruiere optimale „Minimallösungen“ und \textbf{erweitere} diese zu immer \textbf{größeren} Optimallösungen, bis Optimallösung des ursprünglichen Problems zusammensetzbar
		%\pause
		%\item Meistens \textbf{zweidimensional}: Optimal erreichbarer Wert abhängig von betrachtetem Gegenstand \textbf{und} Restbestand
		%\pause
		%\item Formal: DP ist anwendbar, wenn die optimale Lösung aus optimalen Lösungen von Teilproblemen besteht
	\end{itemize}
\end{frame}

\begin{frame}[t]{Optimierung – DP}
	\vspace{-.5\baselineskip}
	\textbf{Lösung von \Knapsack mit DP}  
	\begin{itemize}
		\item Lege zweidim. $\KwArray P[0..n, \ 0..M] \KwOf \R$ an: \\ 
		\impl $P[i, C] = $ \textbf{optimaler Profit} für betrachtete Gegenstände $1...i$ \\ mit benutzter Kapazität $\leq C$ \\
		\impl Rekursionsformel: \\ \vspace{-.7\baselineskip}
		\qqquad \quad $P[i, C] = \max\left(\overbrace{P[i-1, C]}^\text{Nehmen $i$ nicht}, \ \underbrace{P[i-1, C - w_i] + p_i}_\text{Nehmen Gegenstand $i$ mit}\right)$
		\pause 
		\item 
		\lFor{Items $i := 1 \KwTo n$}{\quad \lFor{Capacity $C := 1 \KwTo M$}{}} \vspace{-\baselineskip}
		\quad $P[i, C] := \max\left(\textcolor{darkgreen}{P[i-1, C]}, \ \textcolor{blue}{P[i-1, C - w_i] + p_i}\right)$ \\
		\quad $Taken[i, C] := \KwTrue \KwSty{ if} $ \textcolor{darkgreen}{links} $ < $ \textcolor{blue}{rechts}
		\pause
		\item Alternativer „Pseudo-Pseudocode“: \\
		\lFor{Items $i := 1 \KwTo n$}{\quad \lFor{Capacity $C := 1 \KwTo M$}{}} 
		\vspace{-\baselineskip}
		\quad \lIf{Platz langt $\wedge$ $\text{Profit}(\text{\small Restbestand mit $i$}) > \text{Profit}(\text{\small Rest ohne $i$})$}{} 
		\qquad 		$Taken[i, C] := \KwTrue$ \\
		\quad		$P[i, C] := $ besserer Profit von beiden \quad (wird immer gesetzt)
		  
		
		%\item Für alle Gegenstände $i$ von $1$ bis $n$ und Kapazität $C$ von $1$ bis $M$: \\
		%Falls der Platz ausreicht, überprüfe, ob der Profit durch das Einfügen von $i$ samt optimaler Restplatzauffüllung größer ist als die optimale Restplatzauffüllung ohne $i$ (und passe $P$ an).
	\end{itemize}
\end{frame}

\begin{frame}[t]{Optimierung – DP}
	\vspace{-.5\baselineskip}
	\textbf{Lösung von \Knapsack mit DP} 
	\begin{itemize}
		\implitem Erinnerung: \\ \vspace{-\baselineskip}
		\quad $P[i, C] = \max\left(\overbrace{P[i-1, C]}^\text{Nehmen $i$ nicht}, \ \underbrace{P[i-1, C - w_i] + p_i}_\text{Nehmen Gegenstand $i$ mit}\right)$ 
		\forcenewline
		\forcenewline
		\item Ausfüllen für $i = 0$: Keine Gegenstände \impl Kein Profit: $P[0, \_] := 0$
		\pause
		\item Ausfüllen für $i = 1$: Einfach (immer \textbf{rein}, sobald Platz \textbf{reicht})
		\pause
		\item[...] Rest mit Formel ausfüllen...
		\implitem Am \textbf{Ende}: $P[n, M]$ gibt \textbf{maximalen} Profit an 
		\pause
		\item Item-Menge rekonstruieren: $Taken[i, C]$ \textbf{rückwärts} laufen ab $C := M$ \\
		\lFor{$i := n \KwDownto 1$}{} 
		\quad $NowReallyTaken[i] := Taken[i, C]$ \\
		\quad \lIf{$Taken[i, C]$}{$ C \minuseq w_i$}
		\item \textbf{Gesamt-Laufzeit}: $O(n \cdot M)$, aber \textbf{pseudo}polynomiell 
	\end{itemize}
\end{frame}

\begin{frame}{Optimierung – DP: Exkurs {\footnotesize (Nicht klausurrelevant)}}
	\begin{exampleblock}{Ein haarspaltender Einwurf}
		\begin{algorithm}[H]
			\Function {InsanelyComplicated$(n : \N)$} {
				$sum := 0$\;
				\For{$i := 1 \KwTo n$} {
					$sum\pp$\;
				}
				\KwRet{$sum$}
			}
		\end{algorithm}
	\end{exampleblock}
	Welche Laufzeit hat dieser Algorithmus?
\end{frame}

\begin{frame}{Optimierung – DP: Exkurs {\footnotesize (Nicht klausurrelevant)}}
	\textbf{Laufzeit: Mehr Schein als Sein} 
	\begin{itemize}
		\item \textbf{Eigentlich} heißt „Laufzeit“: Laufzeit in Bezug auf Eingabe\textbf{größe} \\
		{\small („\textit{wieviel} Elemente zum Durchlaufen“ o.~ä.)}
		\pause
		\item Eingabe keine Elemente, sondern ein \textbf{Wert} $n$? \\
		 $n$ wird (meist binär) \textbf{kodiert} in Größe $a := \log n$ \\ 
		 \impl $a$ ist \textbf{tatsächliche} Eingabegröße! \\
		 \impl die eigentliche Laufzeit: $O(n) = O(2^a)$ \impl \textbf{exponentiell}
		\pause
		\item \textbf{Aber}: Laufzeit immerhin polynomiell in Bezug auf {\small (größten)} Eingabe\textbf{wert} $n$ \impl Bezeichnung: \textbf{Pseudo}polynomiell \\
	\end{itemize}
\end{frame}

\begin{frame}{Optimierung – DP {\footnotesize (Nicht klausurrelevant)}}
	\textbf{\Knapsack mit DP: Laufzeit} 
	\begin{itemize}
		\item DP-Algorithmus für \Knapsack: \textbf{Laufzeit} in $O(n \cdot M)$
		\item $n$ Elemente sind „\textbf{echt da}“, aber $M$ ist „irgendein \textbf{Wert}“ \\
		\impl Laufzeit auch \textbf{pseudopolynomiell}
		\pause
		\item \Knapsack ist \textbf{\NP-vollständig}, d.~h. für \Knapsack ist \textbf{kein echt polynomieller} Algo bekannt
		\pause
		\implitem So einer würde die \textbf{große ungelöste Frage} $\mathcal{P} = \NP$ klären \\ 
		(und dem Finder 1~000~000~\$ einbringen \smiley).
		\pause
		\item Mehr dazu in TGI nächstes Semester...
	\end{itemize}
\end{frame}


\begin{frame}{Optimierungsprobleme – ILPs}
	\textbf{(Integer) Linear Programming} \\
	Lineares Programm (LP) \\ mit $n$ Variablen und $m$ Constraints ($=$ Beschränkungen):   % TODO Neuformatieren
	\begin{itemize}
		\pause
		\item \textbf{Lösungsvektor} $x = (x_1, ..., x_n) \in \R_{\geq0}^n$ (wird \textbf{gesucht})
		\pause
		\item \textbf{Kosten}-/\textbf{Gewinnvektor} $c = (c_1, ..., c_n)\in \R^n$; \\ 
		$f(x) = c \cdot x = \sum c_i  x_i $ soll minimiert/maximiert werden
		\pause
		\item $m$ \textbf{Constraints}, \quad für $j = 1...m$: \\ \vspace{.1\baselineskip}
		\quad $a_j \cdot x \braced{\stackedtight{\leq \\ = \\ \geq}} b_j$ \quad mit $a_j = (a_{j1}, ..., a_{jn}) \in \R^n$, \ $b_j \in \R$  % und $\sim_i\ \in \big\lbrace\leq, =, \geq\big\rbrace$
	\end{itemize}
	\pause
	Varianten:
	\begin{itemize}
		\item \textbf{Integer LP}: LP mit allen $x_i \in \N_0$ \\
		(oft durch Constraints auch $x_i \in \{0, 1\}$)
		\pause
		\item \textbf{Mixed ILP}: LP, bei dem \textbf{einige} (aber nicht alle) $x_i$ $\in \N_0$ sind
	\end{itemize}
\end{frame}

\begin{frame}{Optimierungsprobleme – ILPs}
	\textbf{Beispiel: \Knapsack als ILP} 
	\begin{itemize} % TODO Neuformatieren
		\item \textbf{Lösungsvektor} $x \in \{0, 1\}^n$: \\ 
		\quad  $x_i = 1$ \gdw Gegenstand $i$ wird eingepackt
		\pause 
		\item Profitwerte $p_i$ bilden schon Profitvektor $p$: \\ \impl \textbf{Profitfunktion} $f(x) = p \cdot x$ soll \textbf{maximiert} werden.
		\pause
		\item \textbf{Constraints}: Nur einen, nämlich \\
		 \quad $w \cdot x \leq M$ \quad mit $w = (w_1,..,w_n)$
		\pause
		\forcenewline
		\forcenewline
		\item Wie \textbf{lösen} wir das jetzt? \\
		\impl Wir \textbf{gar nicht}, aber ein \textit{Black-Box-Solver} für ILPs schon \smiley
	\end{itemize}
\end{frame}


\begin{frame}{Optimierungsprobleme – ILPs}
	\textbf{Warum dann überhaupt (M)ILPs?} 
	\begin{itemize}
		\item Es gibt viele \textbf{sehr effiziente} Löser für (M)ILPs \\
		\impl \textbf{Relevantes Thema}
		\pause
		\item \textbf{Sehr viele Probleme} können als (M)ILPs formuliert werden
		\pause
		\item \textbf{Vorgeschmack} auf TGI (Reduktionen, \NP-Vollständigkeit)
	\end{itemize}
\end{frame}


\begin{frame}{Optimierungsprobleme – ILPs}
	\textbf{Beispiel: \textsc{VertexCover}} \\
	\textsc{VertexCover}: Haben ungerichteten, zusammenhängenden Graphen $G = (V, E)$, wollen \textbf{minimale} Teilmenge $C \subseteq V$, so dass $\forall \{u, v\} \in E: v \in C$   \hfill  \only<all:2>{\small (\textcolor[rgb]{0,.67,1}{\textbf{Blau}}: Ein mögliches Vertex-Cover $C$)}
	
	\begin{figure}[htp]
		\centering
		\only<all:1>{\includegraphics[height=5cm]{vertexcover1}}
		\only<all:2>{\includegraphics[height=5cm]{vertexcover2}}
	\end{figure}
\end{frame}

\begin{frame}{Optimierungsprobleme – ILPs}
	\textbf{\textsc{VertexCover} als ILP} 
	\begin{itemize} %TODO Formatierung
		\item \textbf{Lösungsvektor} $x \in \{0, 1\}^n$: \quad $x_i = 1$ \gdw Knoten $i \in C$ 
		\pause
		\item Kostenvektor $c = (1,..,1) \in \{1\}^n$, \\ 
		\textbf{minimiere} $f(x) = c \cdot x = \sum x_i = |C|$
		\pause
		\item $m$ \textbf{Constraints}: \quad  $\forall \{u, v\} \in E \quad \text{ jeweils } \quad x_u + x_v \geq 1$
	\end{itemize}
\end{frame}

\iffalse
\begin{frame}<handout:0>{Optimierungsprobleme}
	\textbf{Zusammenfassung: Was haben wir heute gelernt?} 
	\begin{itemize}
		\item Verschiedene Verfahren für das Lösen von Optimierungsproblemen: Greedy, DP, ILP
		\pause
		\item TGI wird bestimmt super
		\pause
		\item Gruppenarbeit ist \textbf{megatoll} und überhaupt das \textbf{AllerBUÄSTE} von der \textbf{ganzön Welt}, deshalb sollten wir \textbf{unbedingt} jetzt gleich eine solche machön!!!1!
	\end{itemize}
\end{frame}
\fi

\begin{frame}{Optimierungsprobleme – ILPs}
	\underline{Aufgabe: } \\
	{\scriptsize  \newcommand{\Actor}{Doktor Meta\ } Dem ebenso verrückten wie vergesslichen Superbösewicht \Actor ist nach langen Nächten der Schlaflosigkeit wieder eingefallen, dass er ja noch die Weltherrschaft erlangen wollte. 
	Als ersten Schachzug zu seinem genialen Triumph möchte er die Kontrolle über seine Heimatstadt Traffalach gewinnen, um sie im Anschluss zur Welthauptstadt zu erklären. 
	Hierzu plant er, sich ein einzigartiges Merkmal von Traffalach zu Nutze machen:
	Als die Stadt gegründet wurde, unterteilte man das Gebiet großflächig in Besitztümer, wobei für jedes Besitztum wiederum mehrere Besitzurkunden unter den Siedlern verteilt wurden. 
	Aus nicht näher bekannten Gründen wurde zudem in der Traffalacher Verfassung festgehalten, dass dem, dem es gelingen sollte, für jedes Besitztum eine der Besitzurkunden zu erlangen, der Besitzanspruch für die gesamte Stadt zufällt. \newline
	Da die mächtigen Herrscherfamilien Traffalachs traditionell untereinander bis aufs Blut verfeindet sind, ist dies bis heute noch niemandem gelungen, doch mit Hilfe eines intriganten Netzwerkes von Unterhändlern konnte \Actor eine Menge von $n$ Angeboten erhalten. 
	Ärgerlicherweise weigern sich seine Geschäftspartner, die Urkunden einzeln zu verkaufen und verlangen stattdessen jeweils eine stattliche Summe von $c_i$ Euro für die Urkundenmenge $U_i$. 
	\Actor hat bereits analysiert, dass die Angebote mehr als ausreichen, um für alle $k$ Besitztümer eine Urkunde zu bekommen. 
	Daher möchte er nun so wenig Geld wie möglich für die Übernahme von Traffalach ausgeben (um möglichst viele Reserven für seinen weiteren Weltherrschafts-Feldzug übrig zu haben). 
	Da er kürzlich gehört hat, was für eine tolle Sache ILPs doch sind, soll ein solches hierbei zum Einsatz kommen, um die Menge der Angebote zu bestimmen, auf die \Actor eingehen sollte.
	
	Formuliert das Problem als ILP.}
\end{frame}

\begin{frame}{Optimierungsprobleme – ILPs}
	\underline{Lösung: } 
	\begin{itemize}
		\item \textbf{Lösungsvektor} $x \in \{0, 1\}^n$, \\ 
		\quad $x_i = 1$ \gdw Urkundenmenge $U_i$ wird gekauft 
		\item \textbf{Kostenvektor} $c = (c_1, ..., c_n) \in \R_{\geq0}^n$ \\
		\quad Minimiere $f(x) = c \cdot x = \sum c_i x_i$
		\item $k$ \textbf{Constraints}, \quad für $j = 1...k$: \\
		\quad $\sum\limits_{i=1}^{n} U_{ij} \cdot x_i  \geq 1$ \quad mit $U_{ij} = \casesl{1, \ \text{Urkunde } j \in U_i \\ 0, \ \text{Urkunde } j \notin U_i}$
	\end{itemize}
	Das Problem heißt allgemein übrigens \textsc{SetCover}.
\end{frame}

\begin{frame}{Schönes Wochenende noch! \smiley}
	\centering
	\textbf{\textsc{Travelling Salesman Problem}} \\[.2\baselineskip]
	\includegraphics[width=.9\textwidth]{xkcd}
\end{frame}

\only<beamer:0>{\slideThanks}

\end{document}