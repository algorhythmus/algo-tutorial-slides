%beamer

%\PassOptionsToClass{handout}{beamer}

\input{../preamble/tutpreamble}

\newcommand{\Knapsack}{\textsc{Knapsack}\xspace}

\begin{document}
	
\starttut{12}
	
\begin{frame}{Schwarzes Brett + Klausurinfos!}
	\begin{itemize}
		%\item {\Large \textbf{\{ \}} vs. \textbf{( )}} – Ungerichtete/gerichtete Kanten
		\item \textbf{Klausur} findet statt am \textbf{04.09.2018} von \textbf{8–10~Uhr} \\
		\item \textbf{Erlaubt}: Stifte, 4-Gänge-Menü, \textbf{Cheatsheet} (1 DIN-A4-Blatt beidseitig beliebig beschrieben) 
		\item Klausur\textbf{anmeldung} bis 28.08.18, 12 Uhr. \\
		Klausur\textbf{abmeldung} bis 28.08.18, 12 Uhr, danach nur \textbf{direkt} vor Klausur im HS!
	\end{itemize}
\end{frame}
	
\sectionheadframe{Optimierungsprobleme}{First World Problems}
	
\begin{frame}{Optimierungsprobleme}
	\textbf{Mehr Effizienz} 
	\begin{itemize}
		\item Dijkstra: \textbf{Kürzeste} Pfade
		\item Jarník-Prim bzw. Kruskal: \textbf{Minimale} Spannbäume
		\item[...]
		\implitem Alles \textbf{Optimierungsprobleme}
		\pause
		\item \textbf{Heute}: Optimierungsprobleme \textbf{allgemein} – und wie man sie \textbf{löst} \\
	\end{itemize}
\end{frame}

\begin{frame}{Optimierungsprobleme – \Knapsack}
	\textbf{Beispiel: Ich nehme meinen Rucksack und packe ein...} 
	\begin{itemize}
		\item \textbf{Rucksackproblem} (\Knapsack): \\
		\textbf{Gegeben}: \\ 
		\quad Rucksackplatz $M$, \\
		\quad $n$ Gegenstände mit \textbf{Gewicht} $w_i$ und \textbf{Profit} $p_i$ \\
		\textbf{Gesucht}: Teilmenge $X$ der Gegenstände, sodass \\
		\quad $\sum\limits_{i \in X} p_i$ \textbf{maximal} wird, aber $\sum\limits_{i \in X} w_i \leq M$ bleibt
		\item \textbf{Nicht alle} Gegenstände passen in den Rucksack
	\end{itemize}
	\forcenewline
	\pause
	Lösungsansätze?
\end{frame}

\section{Greedy-Algorithmen}

\begin{frame}{Optimierung – Greedy}
	\textbf{Wir bereuen nichts: Greedy-Algorithmen} 
	\begin{itemize}
		\item Prinzip: Reine \textbf{Gier}, never step back! \\
		Was \textbf{grad} am \textbf{Besten} scheint: \textbf{Direkt} nehmen!
		\pause
		\implitem Kann in \textbf{Sackgasse} führen 
		\implitem Auf die \textbf{Spitze} geht's manchmal nur durchs \textbf{Tal}
		\pause
		\item Kann aber auch funktionieren: \\
		Dijkstra, Jarník-Prim, Kruskal – alles \textbf{greedy} und läuft \yop
	\end{itemize}
	\pause
	
	Ein \textbf{Greedy-Algo} für \textbf{\Knapsack}:
	\begin{itemize}
		\item Schmeiße der Reihe nach Gegenstände mit \textbf{bestem} Profit-/Gewicht-\textbf{Verhältnis} $\frac{p_i}{w_i}$ rein, bis voll
		\pause
		\implitem Aber: \textbf{nicht optimal}, Bsp.: \ $M=10, \ (p_i, w_i) = (8, 6), (5, 5), (5, 5)$ \crash
		\implitem Greedy-Algorithmus für \Knapsack \textbf{ungeeignet}, kann sich eine optimale Lösung \textbf{verbauen}\\
	\end{itemize}
\end{frame}

\section{Dynamic Programming}

\begin{frame}{Optimierung – DP}
	\textbf{Rekursion rückwärts: Dynamic Programming (DP)} 
	\begin{itemize}
		\item \textbf{Rekursion} aka \textbf{Teile-und-Herrsche}: \\
		Löse großes Problem durch \textbf{Zerlegung} in kleinere
		\pause
		\item \textbf{Dynamic Programming}: \\
		Löse \textbf{Kleinere} zuerst, setze dann zu \textbf{Größeren} zusammen: \\
		\pause
		\quad \hphantom{\~~>} Konstruiere optimale „\textbf{Minimallösungen}“ \\ \quad \~~> zu größeren Optimallösungen \textbf{erweitern} \\ \quad \~~> bis zum urspr. Problem.
		\pause
		\item Meistens \textbf{zweidimensional}: Tabelle mit \textbf{Rekursionsformel} ausfüllen. (siehe Beispiel)
		\pause
		\item Formal: DP ist anwendbar \gdw die optimale Lösung besteht aus optimalen Lösungen von \textbf{Teilproblemen}.
		%\item D.h. konstruiere optimale „Minimallösungen“ und \textbf{erweitere} diese zu immer \textbf{größeren} Optimallösungen, bis Optimallösung des ursprünglichen Problems zusammensetzbar
		%\pause
		%\item Meistens \textbf{zweidimensional}: Optimal erreichbarer Wert abhängig von betrachtetem Gegenstand \textbf{und} Restbestand
		%\pause
		%\item Formal: DP ist anwendbar, wenn die optimale Lösung aus optimalen Lösungen von Teilproblemen besteht
	\end{itemize}
\end{frame}

\newcommand{\fib}{\operatorname{fib}}
\begin{frame}{Optimierung – DP}
	\textbf{Beispiel: Fibonacci-Zahlen} \\
	\begin{itemize}
		\item \§{Definiere }$\fib(0) := \fib(1) := 1, $\\
		\.$\fib(n) := \fib(n-1) + \fib(n-2) \quad \forall n \geq 2$
		\pause
		\item Berechne $\fib(40)$: \\ 
			  $\fib(40) = \fib(39) + \alert{\fib(38)}$, \\
			  $\fib(39) = \alert{\fib(38)} + \fib(37)$ \quad \impl $\alert{\fib(38)}$ wird zweimal berechnet
		\pause
		\implitem $\fib(5)$ wird \realTilde\emph{15-Mio.-mal} berechnet: \textbf{Performance-Katastrophe}! 
		\pause
		\item Besser: Von \textbf{unten nach oben} rechnen und \textbf{Tabelle} ausfüllen \\
		\quad $\fib = (1,1,0\dots0) : \KwArray[0..40] \KwOf \Z$ \\
		\quad \lFor{$i := 2 \KwTo 40$}{} %\\ %WTF LaTeX?
		\qquad $\fib[i] := \fib[i-1] + \fib[i-2]$ \\
		\vspace{.2\baselineskip}
		\mbox{\begin{tabular}{c|c|c|c|c|c}
			$i$       & 0 & 1 & 2 & 3 & ... \\
			\hline
			$\fib[i]$ & 1 & 1 & 2 & 3 & ... 
		\end{tabular}} \quad \Impl
		\implitem „Rekursion rückwärts“
	\end{itemize}
\end{frame}

\begin{frame}[t]{Optimierung – DP}
	\vspace{-.5\baselineskip}
	\textbf{Lösung von \Knapsack mit DP}  
	\begin{itemize}
		\item Lege zweidim. $P: \KwArray[0..n, \ 0..M] \KwOf \R$ an: \\ 
		\impl $P[i, C] = $ \textbf{optimaler Profit} für betrachtete Gegenstände $1...i$ \\ mit benutzter Kapazität $\leq C$ \\
		\impl Rekursionsformel: \\ \vspace{-.7\baselineskip}
		\qqquad \quad $P[i, C] = \max\left(\overbrace{P[i-1, C]}^\text{Nehmen $i$ nicht}, \ \underbrace{P[i-1, C - w_i] + p_i}_\text{Nehmen Gegenstand $i$ mit}\right)$
		\pause 
		\item 
		\lFor{Items $i := 1 \KwTo n$}{\quad \lFor{Capacity $C := 1 \KwTo M$}{}} \vspace{-\baselineskip}
		\quad $P[i, C] := \max\left(\textcolor{darkgreen}{P[i-1, C]}, \ \textcolor{blue}{P[i-1, C - w_i] + p_i}\right)$ \\
		\quad $Taken[i, C] := \tuple{\textcolor{darkgreen}{P[i-1, C]} < \textcolor{blue}{P[i-1, C - w_i] + p_i}} : \|Bool|$
		\pause
		\item Bedeutet \: {\small („Pseudo-pseudo“, so NICHT in der Klausur!)}: \\
		\lFor{Items $i := 1 \KwTo n$}{\quad \lFor{Capacity $C := 1 \KwTo M$}{}} 
		\vspace{-\baselineskip}
		\quad \lIf{Platz langt $\:\wedge\:$ $\text{Profit}(\text{\small Restbestand mit $i$}) > \text{Profit}(\text{\small Rest ohne $i$})$}{} 
		\qquad 		$Taken[i, C] := \KwTrue$ \\
		\quad		$P[i, C] := $ besserer Profit von beiden \quad (wird immer gesetzt)
		  
		
		%\item Für alle Gegenstände $i$ von $1$ bis $n$ und Kapazität $C$ von $1$ bis $M$: \\
		%Falls der Platz ausreicht, überprüfe, ob der Profit durch das Einfügen von $i$ samt optimaler Restplatzauffüllung größer ist als die optimale Restplatzauffüllung ohne $i$ (und passe $P$ an).
	\end{itemize}
\end{frame}

\begin{frame}[t]{Optimierung – DP}
	\vspace{-.5\baselineskip}
	\textbf{Lösung von \Knapsack mit DP} 
	\begin{itemize}
		\implitem Erinnerung: \\ \vspace{-\baselineskip}
		\quad $P[i, C] = \max\left(\overbrace{P[i-1, C]}^\text{Nehmen $i$ nicht}, \ \underbrace{P[i-1, C - w_i] + p_i}_\text{Nehmen Gegenstand $i$ mit}\right)$ 
		\forcenewline
		\forcenewline
		\item Ausfüllen für $i = 0$: Keine Gegenstände \impl Kein Profit: $P[0, \_] := 0$
		\pause
		\item Ausfüllen für $i = 1$: Einfach (immer \textbf{rein}, sobald Platz \textbf{reicht})
		\pause
		\item[...] Rest mit Formel ausfüllen...
		\implitem Am \textbf{Ende}: $P[n, M]$ gibt \textbf{maximalen} Profit an 
		\pause
		\item Item-Menge rekonstruieren: $Taken[i, C]$ \textbf{rückwärts} laufen ab $C := M$ \\
		\lFor{$i := n \KwDownto 1$}{} 
		\quad $NowReallyTaken[i] := Taken[i, C]$ \\
		\quad \lIf{$Taken[i, C]$}{$ C \minuseq w_i$}
		\item \textbf{Gesamt-Laufzeit}: $O(n \cdot M)$, aber \textbf{pseudo}polynomiell 
	\end{itemize}
\end{frame}

\begin{frame}{Optimierung – DP: Exkurs {\footnotesize (Nicht klausurrelevant)}}
	\begin{exampleblock}{Ein haarspaltender Einwurf}
		\begin{algorithm}[H]
			\Function {InsanelyComplicated$(n : \N)$} {
				$sum := 0$\;
				\For{$i := 1 \KwTo n$} {
					$sum\pp$\;
				}
				\KwRet{$sum$}
			}
		\end{algorithm}
	\end{exampleblock}
	Welche Laufzeit hat dieser Algorithmus?
\end{frame}

\begin{frame}{Optimierung – DP: Exkurs {\footnotesize (Nicht klausurrelevant)}}
	\textbf{Laufzeit: Mehr Schein als Sein} 
	\begin{itemize}
		\item \textbf{Eigentlich} heißt „Laufzeit“: Laufzeit in Bezug auf Eingabe\textbf{größe} \\
		{\small („\textit{wieviel} Elemente zum Durchlaufen“ o.~ä.)}
		\pause
		\item Eingabe keine Elemente, sondern ein \textbf{Wert} $n$? \\
		 $n$ wird (meist binär) \textbf{kodiert} in Größe $a := \log n$ \\ 
		 \impl $a$ ist \textbf{tatsächliche} Eingabegröße! \\
		 \impl die eigentliche Laufzeit: $O(n) = O(2^a)$ \impl \textbf{exponentiell}
		\pause
		\item \textbf{Aber}: Laufzeit immerhin polynomiell in Bezug auf {\small (größten)} Eingabe\textbf{wert} $n$ \impl Bezeichnung: \textbf{Pseudo}polynomiell \\
	\end{itemize}
\end{frame}

\begin{frame}{Optimierung – DP {\footnotesize (Nicht klausurrelevant)}}
	\textbf{\Knapsack mit DP: Laufzeit} 
	\begin{itemize}
		\item DP-Algorithmus für \Knapsack: \textbf{Laufzeit} in $O(n \cdot M)$
		\item $n$ Elemente sind „\textbf{echt da}“, aber $M$ ist „irgendein \textbf{Wert}“ \\
		\impl Laufzeit auch \textbf{pseudopolynomiell}
		\pause
		\item \Knapsack ist \textbf{\NP-vollständig}, d.~h. für \Knapsack ist \textbf{kein echt polynomieller} Algo bekannt
		\pause
		\implitem So einer würde die \textbf{große ungelöste Frage} $\mathcal{P} \stackrel{?}{=} \NP$ klären \\ 
		(und dem Finder 1~000~000~\$ einbringen \smiley).
		\pause
		\item Mehr dazu in TGI nächstes Semester...
	\end{itemize}
\end{frame}

\section{Integer Linear Programs}

\begin{frame}{Optimierungsprobleme – ILPs}
	\textbf{(Integer) Linear Programming} \\
	Lineares Programm (LP) \\ mit $n$ Variablen und $m$ Constraints ($=$ Beschränkungen):   % TODO Neuformatieren?
	\begin{itemize}
		\pause
		\item \textbf{Lösungsvektor} $x = (x_1, ..., x_n) \in \R_{\geq0}^n$ (wird \textbf{gesucht})
		\pause
		\item \textbf{Kosten}-/\textbf{Gewinnvektor} $c = (c_1, ..., c_n)\in \R^n$; \\ 
		$f(x) = c \cdot x = \sum c_i  x_i $ soll minimiert/maximiert werden
		\pause
		\item $m$ \textbf{Constraints}, \quad für $j = 1...m$: \\ \vspace{.1\baselineskip}
		\quad $a_j \cdot x \braced{\stackedtight{\leq \\ = \\ \geq}} b_j$ \quad mit $a_j = (a_{j1}, ..., a_{jn}) \in \R^n$, \ $b_j \in \R$  % und $\sim_i\ \in \big\lbrace\leq, =, \geq\big\rbrace$
	\end{itemize}
	\pause
	Varianten:
	\begin{itemize}
		\item \textbf{Integer LP}: LP mit allen $x_i \in \N_0$ \\
		(oft durch Constraints auch $x_i \in \{0, 1\}$)
		\pause
		\item \textbf{Mixed ILP}: LP, bei dem \textbf{einige} (aber nicht alle) $x_i$ $\in \N_0$ sind
	\end{itemize}
\end{frame}

\begin{frame}{Optimierungsprobleme – ILPs}
	\textbf{Beispiel: \Knapsack als ILP} 
	\begin{itemize} 
		\item \textbf{Lösungsvektor} $x \in \{0, 1\}^n$: \\ 
		\quad  $x_i = 1$ \gdw Gegenstand $i$ wird eingepackt
		\pause 
		\item Profitwerte $p_i$ bilden schon Profitvektor $p$: \\ \impl \textbf{Profitfunktion} $f(x) = p \cdot x$ soll \textbf{maximiert} werden.
		\pause
		\item \textbf{Constraints}: Nur einen, nämlich \\
		 \quad $w \cdot x \leq M$ \quad mit $w = (w_1,..,w_n)$
		\pause
		\item In üblicher Schreibweise: 
		\begin{align*}
			\max_{x \in \set{0,1}^n} &f(x) = p \* x \\
			 \text{sodass} \quad & w \* x \leq M \\
			\text{mit} \quad & \text{$w = (w_1,..,w_n)$,\, $p = (p_1,..,p_n)$.} 
		\end{align*} \vspace{-1.2\baselineskip}
		\pause
		\item Wie \textbf{lösen} wir das jetzt? \\
		\impl Wir \textbf{gar nicht}, aber ein \textit{Black-Box-Solver} für ILPs schon \smiley
	\end{itemize}
\end{frame}


\begin{frame}{Optimierungsprobleme – ILPs}
	\textbf{Warum dann überhaupt (M)ILPs?} 
	\begin{itemize}
		\item Es gibt viele \textbf{sehr effiziente} Löser für (M)ILPs \\
		\impl \textbf{Relevantes Thema}
		\pause
		\item \textbf{Sehr viele Probleme} können als (M)ILPs formuliert werden
		\pause
		\item \textbf{Vorgeschmack} auf TGI (Reduktionen, \NP-Vollständigkeit)
		\item {\small (Ein ganzes Ergänzungsfach dazu: Operations Research)}
	\end{itemize}
\end{frame}


\begin{frame}{Optimierungsprobleme – ILPs}
	\textbf{Beispiel: \textsc{VertexCover}} \\
	\textsc{VertexCover}: Haben ungerichteten, zusammenhängenden Graphen $G = (V, E)$, wollen \textbf{minimale} Teilmenge $C \subseteq V$, so dass $\forall \{u, v\} \in E: v \in C$   \hfill  \only<all:2>{\mbox{\small (\textcolor[rgb]{0,.67,1}{\textbf{Blau}}: Ein mögliches Vertex-Cover $C$)}}
	
	\begin{figure}[htp]
		\centering
		\only<all:1>{\includegraphics[height=5cm]{vertexcover1}}
		\only<all:2>{\includegraphics[height=5cm]{vertexcover2}}
	\end{figure}
\end{frame}

\begin{frame}{Optimierungsprobleme – ILPs}
	\textbf{\textsc{VertexCover} als ILP} 
	\begin{itemize}
		\item \textbf{Lösungsvektor} $x \in \{0, 1\}^n$: \quad $x_i = 1$ \gdw Knoten $i \in C$ 
		\pause
		\item Kostenvektor $c = (1,..,1) \in \{1\}^n$, \\ 
		\textbf{minimiere} $f(x) = c \cdot x = \sum x_i = |C|$
		\pause
		\item $m$ \textbf{Constraints}: \quad  $\forall \{u, v\} \in E \quad \text{ jeweils } \quad x_u + x_v \geq 1$
		\pause 
		\begin{align*}
			{\Impl} \qquad \min_{x \in \set{0,1}^n} &f(x) = c \* x \\
			\text{sodass} \quad & x_u + x_v \geq 1 \quad \forall \set{u,v} \in E \\
			\text{mit} \quad & \text{$c = (1,..,1)$.} 
		\end{align*}
	\end{itemize}
\end{frame}

\iffalse
\begin{frame}<handout:0>{Optimierungsprobleme}
	\textbf{Zusammenfassung: Was haben wir heute gelernt?} 
	\begin{itemize}
		\item Verschiedene Verfahren für das Lösen von Optimierungsproblemen: Greedy, DP, ILP
		\pause
		\item TGI wird bestimmt super
		\pause
		\item Gruppenarbeit ist \textbf{megatoll} und überhaupt das \textbf{AllerBUÄSTE} von der \textbf{ganzön Welt}, deshalb sollten wir \textbf{unbedingt} jetzt gleich eine solche machön!!!1!
	\end{itemize}
\end{frame}
\fi

\begin{frame}{Optimierungsprobleme – ILPs}
	\vspace{-.3\baselineskip}
	\taskheading{Seine Abschiedsvorstellung}
	Der ebenso geniale wie wahnsinnige Superbösewicht Doktor Meta lacht hysterisch: Die Weltherrschaft ist zum Greifen nah! Der verrückte Superschurke plant nämlich, die gesamte Welt in einer Nacht in Schutt und Asche zu legen, um anschließend eine neue Weltordnung aufzubauen. Dafür muss er lediglich ausreichend Militärdepots und sonstige explosionsgefährdete Lagerstätten in die Luft jagen. Genügend Schurken und Handlanger zu rekrutieren, die bei der Verlegung von Zündschnüren helfen, kostet allerdings viel Geld. Er will daher möglichst wenig Depots in die Luft sprengen und trotzdem die ganze Welt, also alle Planquadrate $1...k$, ausradieren. Auf der Welt gibt es insgesamt $n$ Depots, wobei ein Depot $i$ die Planquadrate $P_i \subseteq \set{1,...,k}$ dem Erdboden gleichmachen kann, was allerdings $c_i$ Euro kostet. Nun muss der Superbösewicht lediglich die billigste Menge aller Depots bestimmen, die alle Planquadrate $1...k$ abdeckt.  \\
	Helft mit, die Welt zu zerstören und formuliert das Problem als ILP.
\end{frame}

\begin{frame}{Optimierungsprobleme – ILPs}
	\solutionheading
	\begin{itemize}
		\item \textbf{Lösungsvektor} $x \in \{0, 1\}^n$, \\ 
		\§{\quad $x_i = 1$ \gdw }Depot $i$ wird zur Sprengung ausgewählt  \\
		\. (\entspr Planquadrate $P_i$ sind abgedeckt)
		\item \textbf{Kostenvektor} $c = (c_1, ..., c_n) \in \R_{\geq0}^n$ \\
		\quad Minimiere $f(x) = c \cdot x = \sum c_i x_i$
		\item $k$ \textbf{Constraints}, \quad für $j = 1...k$: \\
		\quad $\sum\limits_{i=1}^{n} P_{ij} \cdot x_i  \geq 1$ \quad mit $P_{ij} = \casesl{1, \ \text{Planquadrat } j \in P_i \\ 0, \ \text{Planquadrat } j \notin P_i}$
		\implitem Andere Schreibweise: \\
			$\min\limits_{x \in \set{0,1}^n} c \* x \quad \text{sodass} \quad P^\top \* x \geq \scalebox{.7}{$\large \Matrix{1 \\ \vdots \\ 1}$}$\!\! \quad\! mit $P = (P_{ij})$, $P_{ij} = ...$, $c = ...$ 
	\end{itemize}
	Das Problem heißt allgemein übrigens \textsc{SetCover}.
\end{frame}

\begin{frame}{Danke für eure Aufmerksamkeit! \smiley}
	\centering
	\textbf{\textsc{Travelling Salesman Problem}} \\[.2\baselineskip]
	\includegraphics[width=.9\textwidth]{xkcd}
\end{frame}

\only<beamer:0>{\slideThanks}

\end{document}