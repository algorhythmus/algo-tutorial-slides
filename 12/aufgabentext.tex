 \newcommand{\Actor}{Doktor Meta\ } Dem ebenso verrückten wie vergesslichen Superbösewicht \Actor ist nach langen Nächten der Schlaflosigkeit wieder eingefallen, dass er ja noch die Weltherrschaft erlangen wollte. 
	Als ersten Schachzug zu seinem genialen Triumph möchte er die Kontrolle über seine Heimatstadt Traffalach gewinnen, um sie im Anschluss zur Welthauptstadt zu erklären. 
	Hierzu plant er, sich ein einzigartiges Merkmal von Traffalach zu Nutze machen:
	Als die Stadt gegründet wurde, unterteilte man das Gebiet großflächig in Besitztümer, wobei für jedes Besitztum wiederum mehrere Besitzurkunden unter den Siedlern verteilt wurden. 
	Aus nicht näher bekannten Gründen wurde zudem in der Traffalacher Verfassung festgehalten, dass dem, dem es gelingen sollte, für jedes Besitztum eine der Besitzurkunden zu erlangen, der Besitzanspruch für die gesamte Stadt zufällt. \newline
	Da die mächtigen Herrscherfamilien Traffalachs traditionell untereinander bis aufs Blut verfeindet sind, ist dies bis heute noch niemandem gelungen, doch mit Hilfe eines intriganten Netzwerkes von Unterhändlern konnte \Actor eine Menge von $n$ Angeboten erhalten. 
	Ärgerlicherweise weigern sich seine Geschäftspartner, die Urkunden einzeln zu verkaufen und verlangen stattdessen jeweils eine stattliche Summe von $c_i$ Euro für die Urkundenmenge $U_i$. 
	\Actor hat bereits analysiert, dass die Angebote mehr als ausreichen, um für alle $k$ Besitztümer eine Urkunde zu bekommen. 
	Daher möchte er nun so wenig Geld wie möglich für die Übernahme von Traffalach ausgeben (um möglichst viele Reserven für seinen weiteren Weltherrschafts-Feldzug übrig zu haben). 
	Da er kürzlich gehört hat, was für eine tolle Sache ILPs doch sind, soll ein solches hierbei zum Einsatz kommen, um die Menge der Angebote zu bestimmen, auf die \Actor eingehen sollte.
	
	Formuliert das Problem als ILP.