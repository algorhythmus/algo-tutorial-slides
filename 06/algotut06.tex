%beamer

\input{../preamble/tutpreamble}

\begin{document}
	
\starttut{6}

\iffalse

\begin{frame}{Zum letzten Übungsblatt}
	\textbf{Achtung bei der Wahl von Datenstrukturen} \\[0,125cm]
	\begin{itemize}
		\pause
		\item Aufgabenstellung spezifiziert nicht, ob eine Menge von Daten als Array oder Liste gegeben ist \impl Beides darf als Parametertyp gewählt werden
		\pause
		\item Aber: Wenn über die Daten iteriert wird, folgendes beachten:
		\pause
		\item Listen bieten Indexzugriff i.A. \textbf{nicht in konstanter Zeit}! D.h. wenn man „via Index“ über eine Liste iteriert, hat die Iteration eine Laufzeit von $\Theta(n^2)$
		\pause
		\item Daher: Iteration bei Arrays frei via Index oder \textit{for-each}, bei Listen \textbf{nur} \textit{for-each}, wenn die Laufzeit wichtig ist (bei Arrays und Listen kann davon ausgegangen werden, dass \textit{for-each} die Daten in linearer Reihenfolge durchläuft)
	\end{itemize}
\end{frame}

\fi

\begin{frame}{Quicksort}
	\textbf{Eine erquickende Neuerung} \\[0,125cm]
	\begin{itemize}
		\item Erinnerung: Array sortierbar durch Einteilung in sortierten und unsortierten Bereich
		\pause
		\implitem \textbf{Idee: „Semi-Sortierung“} \\
		Wähle beliebiges Pivotelement $p$ (in $O(1)$) und teile auf (in $O(n)$):
		\begin{tabular}{|c|c|}
			\hline
			$\qquad \leq p \qquad$  & $\qquad > p \qquad$  \\
			\hline
		\end{tabular} \\
		\forcenewline
		Diese Teile dann rekursiv weitersortieren.
	\end{itemize}
\end{frame}

\begin{frame}{Quicksort – Beispiel}
	Sortiere $A = (5,3,8,4,2,6,1): \KwArray[1..n] \KwOf \N$ mit Quicksort. Wähle als Pivot $p(A) := A[\ceil{\fract n/2 }]$. Zeichne dazu den Rekursionsbaum.
	\visible<2->{
		\begin{center}
			\includegraphics[width=.5\textwidth]{qstree}
		\end{center}	
	}
\end{frame}

\begin{frame}{Quicksort (mit partition)}
	\begin{itemize}
		\item Wie effizient und platzsparend aufteilen? \\
		\impl \textbf{partition}! $O(1)$ Platz und $O(n)$ Zeit  \\
		(Siehe nächste Folien...)
		\pause
		\item \textbf{Laufzeit}: Master-Theorem nicht anwendbar, da Größe der rekursiven Aufrufe \textbf{nicht} in Voraus bekannt
		\pause
		\item Worst-Case $\Theta(n^2)$ möglich
		\pause
		\item Vorlesung sagt: \textbf{Erwartete} Laufzeit in $\Theta(n \log n)$
	\end{itemize}
\end{frame}

\begin{frame}[t]{Quicksort (mit partition)}
	\textbf{Beispiel} \\
	Partitioniere $A: \KwArray[0..n-1]$ mit Pivotwahl $p(A) := A[\floor{\fract n/3 }]$ {\small (mit $n := \abs{A}$)}
	\\[0,5cm]
	\begin{tabular}{ | c | c | c | c | c | c | c | }
		\multicolumn{7}{ c }{ }
		\\ \hline
		1 & 8 & 6 & 9 & 1 & 7 & 0
		\\ \hline
	\end{tabular}\\
	\vspace{3\baselineskip}
	\hyperlink{label:afterEx1}{Hier klicken, um das Beispiel zu überspringen.}
\end{frame}



\begin{frame}[t]{Quicksort (mit partition)}
	\textbf{Beispiel} \\
	Partitioniere $A: \KwArray[0..n-1]$ mit Pivotwahl $p(A) := A[\floor{\fract n/3 }]$ {\small (mit $n := \abs{A}$)}
	\\[0,5cm]
	\begin{tabular}{ | c | c | c | c | c | c | c | }
		\multicolumn{2}{ c| }{ } & p & \multicolumn{4}{ c }{ }
		\\ \hline
		1 & 8 & \cellcolor{yellow} 6 & 9 & 1 & 7 & 0
		\\ \hline
	\end{tabular}
\end{frame}

\begin{frame}[t]{Quicksort (mit partition)}
	\textbf{Beispiel} \\
	Partitioniere $A: \KwArray[0..n-1]$ mit Pivotwahl $p(A) := A[\floor{\fract n/3 }]$ {\small (mit $n := \abs{A}$)}
	\\[0,5cm]
	\begin{tabular}{ | c | c | c | c | c | c | c | }
		\multicolumn{6}{ c| }{ } & p
		\\ \hline
		1 & 8 & 0 & 9 & 1 & 7 & \cellcolor{yellow} 6
		\\ \hline
	\end{tabular}
\end{frame}

\begin{frame}[t]{Quicksort (mit partition)}
	\textbf{Beispiel} \\
	Partitioniere $A: \KwArray[0..n-1]$ mit Pivotwahl $p(A) := A[\floor{\fract n/3 }]$ {\small (mit $n := \abs{A}$)}
	\\[0,5cm]
	\begin{tabular}{ | c | c | c | c | c | c | c | }
		\multicolumn{6}{ c| }{ } & p
		\\ \hline
		\cellcolor{cyan!50} 1 & 8 & 0 & 9 & 1 & 7 & \cellcolor{yellow} 6
		\\ \hline
	\end{tabular}
\end{frame}

\begin{frame}[t]{Quicksort (mit partition)}
	\textbf{Beispiel} \\
	Partitioniere $A: \KwArray[0..n-1]$ mit Pivotwahl $p(A) := A[\floor{\fract n/3 }]$ {\small (mit $n := \abs{A}$)}
	\\[0,5cm]
	\begin{tabular}{ | c | c | c | c | c | c | c | }
		$ \leq p$ & \multicolumn{5}{ c| }{ } & p
		\\ \hline
		\cellcolor{adaptingred} 1 & 8 & 0 & 9 & 1 & 7 & \cellcolor{yellow} 6
		\\ \hline
	\end{tabular}
\end{frame}

\begin{frame}[t]{Quicksort (mit partition)}
	\textbf{Beispiel} \\
	Partitioniere $A: \KwArray[0..n-1]$ mit Pivotwahl $p(A) := A[\floor{\fract n/3 }]$ {\small (mit $n := \abs{A}$)}
	\\[0,5cm]
	\begin{tabular}{ | c | c | c | c | c | c | c | }
		$ \leq p$ & \multicolumn{5}{ c| }{ } & p
		\\ \hline
		\cellcolor{adaptingred} 1 & \cellcolor{cyan!50} 8 & 0 & 9 & 1 & 7 & \cellcolor{yellow} 6
		\\ \hline
	\end{tabular}
\end{frame}

\begin{frame}[t]{Quicksort (mit partition)}
	\textbf{Beispiel} \\
	Partitioniere $A: \KwArray[0..n-1]$ mit Pivotwahl $p(A) := A[\floor{\fract n/3 }]$ {\small (mit $n := \abs{A}$)}
	\\[0,5cm]
	\begin{tabular}{ | c | c | c | c | c | c | c | }
		$ \leq p$ & $ > p$ & \multicolumn{4}{ c| }{ } & p
		\\ \hline
		\cellcolor{adaptingred} 1 & \cellcolor{adaptinggreen} 8 & 0 & 9 & 1 & 7 & \cellcolor{yellow} 6
		\\ \hline
	\end{tabular}
\end{frame}

\begin{frame}[t]{Quicksort (mit partition)}
	\textbf{Beispiel} \\
	Partitioniere $A: \KwArray[0..n-1]$ mit Pivotwahl $p(A) := A[\floor{\fract n/3 }]$ {\small (mit $n := \abs{A}$)}
	\\[0,5cm]
	\begin{tabular}{ | c | c | c | c | c | c | c | }
		$ \leq p$ & $ > p$ & \multicolumn{4}{ c| }{ } & p
		\\ \hline
		\cellcolor{adaptingred} 1 & \cellcolor{adaptinggreen} 8 & \cellcolor{cyan!50} 0 & 9 & 1 & 7 & \cellcolor{yellow} 6
		\\ \hline
	\end{tabular}
\end{frame}

\begin{frame}[t]{Quicksort (mit partition)}
	\textbf{Beispiel} \\
	Partitioniere $A: \KwArray[0..n-1]$ mit Pivotwahl $p(A) := A[\floor{\fract n/3 }]$ {\small (mit $n := \abs{A}$)}
	\\[0,5cm]
	\begin{tabular}{ | c | c | c | c | c | c | c | }
		\multicolumn{2}{ |c| }{$ \leq p$} & $ > p$ & \multicolumn{3}{ c| }{ } & p
		\\ \hline
		\cellcolor{adaptingred} 1 & \cellcolor{adaptingred} 0 & \cellcolor{adaptinggreen} 8 & 9 & 1 & 7 & \cellcolor{yellow} 6
		\\ \hline
	\end{tabular}
\end{frame}

\begin{frame}[t]{Quicksort (mit partition)}
	\textbf{Beispiel} \\
	Partitioniere $A: \KwArray[0..n-1]$ mit Pivotwahl $p(A) := A[\floor{\fract n/3 }]$ {\small (mit $n := \abs{A}$)}
	\\[0,5cm]
	\begin{tabular}{ | c | c | c | c | c | c | c | }
		\multicolumn{2}{ |c| }{$ \leq p$} & $ > p$ & \multicolumn{3}{ c| }{ } & p
		\\ \hline
		\cellcolor{adaptingred} 1 & \cellcolor{adaptingred} 0 & \cellcolor{adaptinggreen} 8 & \cellcolor{cyan!50} 9 & 1 & 7 & \cellcolor{yellow} 6
		\\ \hline
	\end{tabular}
\end{frame}

\begin{frame}[t]{Quicksort (mit partition)}
	\textbf{Beispiel} \\
	Partitioniere $A: \KwArray[0..n-1]$ mit Pivotwahl $p(A) := A[\floor{\fract n/3 }]$ {\small (mit $n := \abs{A}$)}
	\\[0,5cm]
	\begin{tabular}{ | c | c | c | c | c | c | c | }
		\multicolumn{2}{ |c| }{$ \leq p$} & \multicolumn{2}{ c| }{$ > p$} & \multicolumn{2}{ c| }{ } & p
		\\ \hline
		\cellcolor{adaptingred} 1 & \cellcolor{adaptingred} 0 & \cellcolor{adaptinggreen} 8 & \cellcolor{adaptinggreen} 9 & 1 & 7 & \cellcolor{yellow} 6
		\\ \hline
	\end{tabular}
\end{frame}

\begin{frame}[t]{Quicksort (mit partition)}
	\textbf{Beispiel} \\
	Partitioniere $A: \KwArray[0..n-1]$ mit Pivotwahl $p(A) := A[\floor{\fract n/3 }]$ {\small (mit $n := \abs{A}$)}
	\\[0,5cm]
	\begin{tabular}{ | c | c | c | c | c | c | c | }
		\multicolumn{2}{ |c| }{$ \leq p$} & \multicolumn{2}{ c| }{$ > p$} & \multicolumn{2}{ c| }{ } & p
		\\ \hline
		\cellcolor{adaptingred} 1 & \cellcolor{adaptingred} 0 & \cellcolor{adaptinggreen} 8 & \cellcolor{adaptinggreen} 9 & \cellcolor{cyan!50} 1 & 7 & \cellcolor{yellow} 6
		\\ \hline
	\end{tabular}
\end{frame}

\begin{frame}[t]{Quicksort (mit partition)}
	\textbf{Beispiel} \\
	Partitioniere $A: \KwArray[0..n-1]$ mit Pivotwahl $p(A) := A[\floor{\fract n/3 }]$ {\small (mit $n := \abs{A}$)}
	\\[0,5cm]
	\begin{tabular}{ | c | c | c | c | c | c | c | }
		\multicolumn{3}{ |c| }{$ \leq p$} & \multicolumn{2}{ c| }{$ > p$} &  & p
		\\ \hline
		\cellcolor{adaptingred} 1 & \cellcolor{adaptingred} 0 & \cellcolor{adaptingred} 1 & \cellcolor{adaptinggreen} 9 & \cellcolor{adaptinggreen} 8 & 7 & \cellcolor{yellow} 6
		\\ \hline
	\end{tabular}
\end{frame}

\begin{frame}[t]{Quicksort (mit partition)}
	\textbf{Beispiel} \\
	Partitioniere $A: \KwArray[0..n-1]$ mit Pivotwahl $p(A) := A[\floor{\fract n/3 }]$ {\small (mit $n := \abs{A}$)}
	\\[0,5cm]
	\begin{tabular}{ | c | c | c | c | c | c | c | }
		\multicolumn{3}{ |c| }{$ \leq p$} & \multicolumn{2}{ c| }{$ > p$} &  & p
		\\ \hline
		\cellcolor{adaptingred} 1 & \cellcolor{adaptingred} 0 & \cellcolor{adaptingred} 1 & \cellcolor{adaptinggreen} 9 & \cellcolor{adaptinggreen} 8 & \cellcolor{cyan!50} 7 & \cellcolor{yellow} 6
		\\ \hline
	\end{tabular}
\end{frame}

\begin{frame}[t]{Quicksort (mit partition)}
	\textbf{Beispiel} \\
	Partitioniere $A: \KwArray[0..n-1]$ mit Pivotwahl $p(A) := A[\floor{\fract n/3 }]$ {\small (mit $n := \abs{A}$)}
	\\[0,5cm]
	\begin{tabular}{ | c | c | c | c | c | c | c | }
		\multicolumn{3}{ |c| }{$ \leq p$} & \multicolumn{3}{ c| }{$ > p$} & p
		\\ \hline
		\cellcolor{adaptingred} 1 & \cellcolor{adaptingred} 0 & \cellcolor{adaptingred} 1 & \cellcolor{adaptinggreen} 9 & \cellcolor{adaptinggreen} 8 & \cellcolor{adaptinggreen} 7 & \cellcolor{yellow} 6
		\\ \hline
	\end{tabular}
\end{frame}



\begin{frame}[t]{{\hypertarget{label:afterEx1}{}Quicksort (mit partition)}}
	\textbf{Beispiel} \\
	Partitioniere $A: \KwArray[0..n-1]$ mit Pivotwahl $p(A) := A[\floor{\fract n/3 }]$ {\small (mit $n := \abs{A}$)}
	\\[0,5cm]
	\begin{tabular}{ | c | c | c | c | c | c | c | }
		\multicolumn{3}{ |c| }{$ \leq p$} & p & \multicolumn{3}{ c| }{$ > p$}
		\\ \hline
		\cellcolor{adaptingred} 1 & \cellcolor{adaptingred} 0 & \cellcolor{adaptingred} 1 & \cellcolor{yellow} 6 & \cellcolor{adaptinggreen} 8 & \cellcolor{adaptinggreen} 7 & \cellcolor{adaptinggreen} 9
		\\ \hline
	\end{tabular}
\end{frame}



\begin{frame}{Quicksort (mit partition)}
	\textbf{Schema aus der Vorlesung}
	\begin{figure}[htp]
		\centering
		\includegraphics[height=6cm]{quicksortpartition}
	\end{figure}
\end{frame}

\begin{frame}[t]{Quicksort (mit partition)}
	Laufzeit, wenn alle Zahlen gleich sind? \\
	\pause
	\impl $\Theta(n^2)$ \\
	\pause
	\FalseQuestionE{Quicksort (mit partition) ist stabil.}{„Durcheinandermischen“ bei partition macht's kaputt.} 
	\DependsQuestionE{Quicksort ist in-place.}{
		\textbf{Rekursionsaufrufe} benötigen $\Theta( \log n)$ (vernachlässigbar) viel Platz (\impl Stack-Overhead). Abgesehen davon \textbf{kein} weiterer Verwaltungsaufwand.
	}
\end{frame}

\begin{frame}{Quicksort (besseres partition)}
	\textbf{Aller guten Dinge sind drei!} \\
	\begin{itemize}
		\item Worst-Case von eben: schlecht \frownie
		\pause
		\implitem Besser: \textbf{Drei-Wege-Partitionierung}!
		\item Führe einen zusätzlichen Bereich $ = p$ ein: \\
		\begin{tabular}{|c|c|c|}
			\hline
			$\qquad < p \qquad$ & \cellcolor{cyan!50} $\qquad = p \qquad$ & $\qquad > p \qquad$ \\
			\hline
		\end{tabular} 
	\end{itemize}
\end{frame}


\begin{frame}[t]{{{\vspace{.3\baselineskip}Quicksort (mit Drei-Wege-Partitionierung)}}}
	\textbf{Beispiel} \\
	Partitioniere $A: \KwArray[0..n-1]$ mit Pivotwahl $p(A) := A[\floor{\fract n/3 }]$ {\small (mit $n := \abs{A}$)}
	\\[0,5cm]
	\begin{tabular}{ | c | c | c | c | c | c | c | c | c | }
		\multicolumn{9}{ c }{ }
		\\ \hline
		3 & 7 & 0 & 5 & 1 & 5 & 5 & 7 & 1
		\\ \hline
	\end{tabular}\\
	\vspace{3\baselineskip}
	\hyperlink{label:afterEx2}{Hier klicken, um das Beispiel zu überspringen.}
\end{frame}



\begin{frame}[t]{{\vspace{.3\baselineskip}Quicksort (mit Drei-Wege-Partitionierung)}}
	\textbf{Beispiel} \\
	Partitioniere $A: \KwArray[0..n-1]$ mit Pivotwahl $p(A) := A[\floor{\fract n/3 }]$ {\small (mit $n := \abs{A}$)}
	\\[0,5cm]
	\begin{tabular}{ | c | c | c | c | c | c | c | c | c | }
		\multicolumn{9}{ c }{ }
		\\ \hline
		3 & 7 & 0 & \cellcolor{yellow} 5 & 1 & 5 & 5 & 7 & 1
		\\ \hline
	\end{tabular}
\end{frame}

\begin{frame}[t]{{\vspace{.3\baselineskip}Quicksort (mit Drei-Wege-Partitionierung)}}
	\textbf{Beispiel} \\
	Partitioniere $A: \KwArray[0..n-1]$ mit Pivotwahl $p(A) := A[\floor{\fract n/3 }]$ {\small (mit $n := \abs{A}$)}
	\\[0,5cm]
	\begin{tabular}{ | c | c | c | c | c | c | c | c | c | }
		\multicolumn{8}{ c| }{ } & p
		\\ \hline
		3 & 7 & 0 & 1 & 1 & 5 & 5 & 7 & \cellcolor{yellow} 5
		\\ \hline
	\end{tabular}
\end{frame}

\begin{frame}[t]{{\vspace{.3\baselineskip}Quicksort (mit Drei-Wege-Partitionierung)}}
	\textbf{Beispiel} \\
	Partitioniere $A: \KwArray[0..n-1]$ mit Pivotwahl $p(A) := A[\floor{\fract n/3 }]$ {\small (mit $n := \abs{A}$)}
	\\[0,5cm]
	\begin{tabular}{ | c | c | c | c | c | c | c | c | c | }
		\multicolumn{8}{ c| }{ } & p
		\\ \hline
		\cellcolor{cyan!50} 3 & 7 & 0 & 1 & 1 & 5 & 5 & 7 & \cellcolor{yellow} 5
		\\ \hline
	\end{tabular}
\end{frame}

\begin{frame}[t]{{\vspace{.3\baselineskip}Quicksort (mit Drei-Wege-Partitionierung)}}
	\textbf{Beispiel} \\
	Partitioniere $A: \KwArray[0..n-1]$ mit Pivotwahl $p(A) := A[\floor{\fract n/3 }]$ {\small (mit $n := \abs{A}$)}
	\\[0,5cm]
	\begin{tabular}{ | c | c | c | c | c | c | c | c | c | }
		$ < p$ & \multicolumn{7}{ c| }{ } & p
		\\ \hline
		\cellcolor{adaptingred} 3 & 7 & 0 & 1 & 1 & 5 & 5 & 7 & \cellcolor{yellow} 5
		\\ \hline
	\end{tabular}
\end{frame}

\begin{frame}[t]{{\vspace{.3\baselineskip}Quicksort (mit Drei-Wege-Partitionierung)}}
	\textbf{Beispiel} \\
	Partitioniere $A: \KwArray[0..n-1]$ mit Pivotwahl $p(A) := A[\floor{\fract n/3 }]$ {\small (mit $n := \abs{A}$)}
	\\[0,5cm]
	\begin{tabular}{ | c | c | c | c | c | c | c | c | c | }
		$ < p$ & \multicolumn{7}{ c| }{ } & p
		\\ \hline
		\cellcolor{adaptingred} 3 & \cellcolor{cyan!50} 7 & 0 & 1 & 1 & 5 & 5 & 7 & \cellcolor{yellow} 5
		\\ \hline
	\end{tabular}
\end{frame}

\begin{frame}[t]{{\vspace{.3\baselineskip}Quicksort (mit Drei-Wege-Partitionierung)}}
	\textbf{Beispiel} \\
	Partitioniere $A: \KwArray[0..n-1]$ mit Pivotwahl $p(A) := A[\floor{\fract n/3 }]$ {\small (mit $n := \abs{A}$)}
	\\[0,5cm]
	\begin{tabular}{ | c | c | c | c | c | c | c | c | c | }
		$ < p$ & $ > p $ & \multicolumn{6}{ c| }{ } & p
		\\ \hline
		\cellcolor{adaptingred} 3 & \cellcolor{adaptinggreen} 7 & 0 & 1 & 1 & 5 & 5 & 7 & \cellcolor{yellow} 5
		\\ \hline
	\end{tabular}
\end{frame}

\begin{frame}[t]{{\vspace{.3\baselineskip}Quicksort (mit Drei-Wege-Partitionierung)}}
	\textbf{Beispiel} \\
	Partitioniere $A: \KwArray[0..n-1]$ mit Pivotwahl $p(A) := A[\floor{\fract n/3 }]$ {\small (mit $n := \abs{A}$)}
	\\[0,5cm]
	\begin{tabular}{ | c | c | c | c | c | c | c | c | c | }
		$ < p$ & $ > p $ & \multicolumn{6}{ c| }{ } & p
		\\ \hline
		\cellcolor{adaptingred} 3 & \cellcolor{adaptinggreen} 7 & \cellcolor{cyan!50} 0 & 1 & 1 & 5 & 5 & 7 & \cellcolor{yellow} 5
		\\ \hline
	\end{tabular}
\end{frame}

\begin{frame}[t]{{\vspace{.3\baselineskip}Quicksort (mit Drei-Wege-Partitionierung)}}
	\textbf{Beispiel} \\
	Partitioniere $A: \KwArray[0..n-1]$ mit Pivotwahl $p(A) := A[\floor{\fract n/3 }]$ {\small (mit $n := \abs{A}$)}
	\\[0,5cm]
	\begin{tabular}{ | c | c | c | c | c | c | c | c | c | }
		\multicolumn{2}{ |c| }{$ < p$} & $ > p $ & \multicolumn{5}{ c| }{ } & p
		\\ \hline
		\cellcolor{adaptingred} 3 & \cellcolor{adaptingred} 0 & \cellcolor{adaptinggreen} 7 & 1 & 1 & 5 & 5 & 7 & \cellcolor{yellow} 5
		\\ \hline
	\end{tabular}
\end{frame}

\begin{frame}[t]{{\vspace{.3\baselineskip}Quicksort (mit Drei-Wege-Partitionierung)}}
	\textbf{Beispiel} \\
	Partitioniere $A: \KwArray[0..n-1]$ mit Pivotwahl $p(A) := A[\floor{\fract n/3 }]$ {\small (mit $n := \abs{A}$)}
	\\[0,5cm]
	\begin{tabular}{ | c | c | c | c | c | c | c | c | c | }
		\multicolumn{2}{ |c| }{$ < p$} & $ > p $ & \multicolumn{5}{ c| }{ } & p
		\\ \hline
		\cellcolor{adaptingred} 3 & \cellcolor{adaptingred} 0 & \cellcolor{adaptinggreen} 7 & \cellcolor{cyan!50} 1 & 1 & 5 & 5 & 7 & \cellcolor{yellow} 5
		\\ \hline
	\end{tabular}
\end{frame}

\begin{frame}[t]{{\vspace{.3\baselineskip}Quicksort (mit Drei-Wege-Partitionierung)}}
	\textbf{Beispiel} \\
	Partitioniere $A: \KwArray[0..n-1]$ mit Pivotwahl $p(A) := A[\floor{\fract n/3 }]$ {\small (mit $n := \abs{A}$)}
	\\[0,5cm]
	\begin{tabular}{ | c | c | c | c | c | c | c | c | c | }
		\multicolumn{3}{ |c| }{$ < p$} & $ > p $ & \multicolumn{4}{ c| }{ } & p
		\\ \hline
		\cellcolor{adaptingred} 3 & \cellcolor{adaptingred} 0 & \cellcolor{adaptingred} 1 & \cellcolor{adaptinggreen} 7 & 1 & 5 & 5 & 7 & \cellcolor{yellow} 5
		\\ \hline
	\end{tabular}
\end{frame}

\begin{frame}[t]{{\vspace{.3\baselineskip}Quicksort (mit Drei-Wege-Partitionierung)}}
	\textbf{Beispiel} \\
	Partitioniere $A: \KwArray[0..n-1]$ mit Pivotwahl $p(A) := A[\floor{\fract n/3 }]$ {\small (mit $n := \abs{A}$)}
	\\[0,5cm]
	\begin{tabular}{ | c | c | c | c | c | c | c | c | c | }
		\multicolumn{3}{ |c| }{$ < p$} & $ > p $ & \multicolumn{4}{ c| }{ } & p
		\\ \hline
		\cellcolor{adaptingred} 3 & \cellcolor{adaptingred} 0 & \cellcolor{adaptingred} 1 & \cellcolor{adaptinggreen} 7 & \cellcolor{cyan!50} 1 & 5 & 5 & 7 & \cellcolor{yellow} 5
		\\ \hline
	\end{tabular}
\end{frame}

\begin{frame}[t]{{\vspace{.3\baselineskip}Quicksort (mit Drei-Wege-Partitionierung)}}
	\textbf{Beispiel} \\
	Partitioniere $A: \KwArray[0..n-1]$ mit Pivotwahl $p(A) := A[\floor{\fract n/3 }]$ {\small (mit $n := \abs{A}$)}
	\\[0,5cm]
	\begin{tabular}{ | c | c | c | c | c | c | c | c | c | }
		\multicolumn{4}{ |c| }{$ < p$} & $ > p $ & \multicolumn{3}{ c| }{ } & p
		\\ \hline
		\cellcolor{adaptingred} 3 & \cellcolor{adaptingred} 0 & \cellcolor{adaptingred} 1 & \cellcolor{adaptingred} 1 & \cellcolor{adaptinggreen} 7 & 5 & 5 & 7 & \cellcolor{yellow} 5
		\\ \hline
	\end{tabular}
\end{frame}

\begin{frame}[t]{{\vspace{.3\baselineskip}Quicksort (mit Drei-Wege-Partitionierung)}}
	\textbf{Beispiel} \\
	Partitioniere $A: \KwArray[0..n-1]$ mit Pivotwahl $p(A) := A[\floor{\fract n/3 }]$ {\small (mit $n := \abs{A}$)}
	\\[0,5cm]
	\begin{tabular}{ | c | c | c | c | c | c | c | c | c | }
		\multicolumn{4}{ |c| }{$ < p$} & $ > p $ & \multicolumn{3}{ c| }{ } & p
		\\ \hline
		\cellcolor{adaptingred} 3 & \cellcolor{adaptingred} 0 & \cellcolor{adaptingred} 1 & \cellcolor{adaptingred} 1 & \cellcolor{adaptinggreen} 7 & \cellcolor{cyan!50} 5 & 5 & 7 & \cellcolor{yellow} 5
		\\ \hline
	\end{tabular}
\end{frame}

\begin{frame}[t]{{\vspace{.3\baselineskip}Quicksort (mit Drei-Wege-Partitionierung)}}
	\textbf{Beispiel} \\
	Partitioniere $A: \KwArray[0..n-1]$ mit Pivotwahl $p(A) := A[\floor{\fract n/3 }]$ {\small (mit $n := \abs{A}$)}
	\\[0,5cm]
	\begin{tabular}{ | c | c | c | c | c | c | c | c | c | }
		\multicolumn{4}{ |c| }{$ < p$} & $ > p $ & \multicolumn{2}{ c| }{ } & \multicolumn{2}{ c| }{ $ = p $}
		\\ \hline
		\cellcolor{adaptingred} 3 & \cellcolor{adaptingred} 0 & \cellcolor{adaptingred} 1 & \cellcolor{adaptingred} 1 & \cellcolor{adaptinggreen} 7 & 7 & 5 & \cellcolor{yellow} 5 & \cellcolor{yellow} 5
		\\ \hline
	\end{tabular}
\end{frame}

\begin{frame}[t]{{\vspace{.3\baselineskip}Quicksort (mit Drei-Wege-Partitionierung)}}
	\textbf{Beispiel} \\
	Partitioniere $A: \KwArray[0..n-1]$ mit Pivotwahl $p(A) := A[\floor{\fract n/3 }]$ {\small (mit $n := \abs{A}$)}
	\\[0,5cm]
	\begin{tabular}{ | c | c | c | c | c | c | c | c | c | }
		\multicolumn{4}{ |c| }{$ < p$} & $ > p $ & \multicolumn{2}{ c| }{ } & \multicolumn{2}{ c| }{ $ = p $}
		\\ \hline
		\cellcolor{adaptingred} 3 & \cellcolor{adaptingred} 0 & \cellcolor{adaptingred} 1 & \cellcolor{adaptingred} 1 & \cellcolor{adaptinggreen} 7 & \cellcolor{cyan!50} 7 & 5 & \cellcolor{yellow} 5 & \cellcolor{yellow} 5
		\\ \hline
	\end{tabular}
\end{frame}

\begin{frame}[t]{{\vspace{.3\baselineskip}Quicksort (mit Drei-Wege-Partitionierung)}}
	\textbf{Beispiel} \\
	Partitioniere $A: \KwArray[0..n-1]$ mit Pivotwahl $p(A) := A[\floor{\fract n/3 }]$ {\small (mit $n := \abs{A}$)}
	\\[0,5cm]
	\begin{tabular}{ | c | c | c | c | c | c | c | c | c | }
		\multicolumn{4}{ |c| }{$ < p$} & \multicolumn{2}{ c| }{ $ > p $} &  & \multicolumn{2}{ c| }{ $ = p $}
		\\ \hline
		\cellcolor{adaptingred} 3 & \cellcolor{adaptingred} 0 & \cellcolor{adaptingred} 1 & \cellcolor{adaptingred} 1 & \cellcolor{adaptinggreen} 7 & \cellcolor{adaptinggreen} 7 & 5 & \cellcolor{yellow} 5 & \cellcolor{yellow} 5
		\\ \hline
	\end{tabular}
\end{frame}

\begin{frame}[t]{{\vspace{.3\baselineskip}Quicksort (mit Drei-Wege-Partitionierung)}}
	\textbf{Beispiel} \\
	Partitioniere $A: \KwArray[0..n-1]$ mit Pivotwahl $p(A) := A[\floor{\fract n/3 }]$ {\small (mit $n := \abs{A}$)}
	\\[0,5cm]
	\begin{tabular}{ | c | c | c | c | c | c | c | c | c | }
		\multicolumn{4}{ |c| }{$ < p$} & \multicolumn{2}{ c| }{ $ > p $} &  & \multicolumn{2}{ c| }{ $ = p $}
		\\ \hline
		\cellcolor{adaptingred} 3 & \cellcolor{adaptingred} 0 & \cellcolor{adaptingred} 1 & \cellcolor{adaptingred} 1 & \cellcolor{adaptinggreen} 7 & \cellcolor{adaptinggreen} 7 & \cellcolor{cyan!50} 5 & \cellcolor{yellow} 5 & \cellcolor{yellow} 5
		\\ \hline
	\end{tabular}
\end{frame}

\begin{frame}[t]{{\vspace{.3\baselineskip}Quicksort (mit Drei-Wege-Partitionierung)}}
	\textbf{Beispiel} \\
	Partitioniere $A: \KwArray[0..n-1]$ mit Pivotwahl $p(A) := A[\floor{\fract n/3 }]$ {\small (mit $n := \abs{A}$)}
	\\[0,5cm]
	\begin{tabular}{ | c | c | c | c | c | c | c | c | c | }
		\multicolumn{4}{ |c| }{$ < p$}  &\multicolumn{2}{ c| }{ $ > p $} & \multicolumn{3}{ c| }{ $ = p $}
		\\ \hline
		\cellcolor{adaptingred} 3 & \cellcolor{adaptingred} 0 & \cellcolor{adaptingred} 1 & \cellcolor{adaptingred} 1 & \cellcolor{adaptinggreen} 7 & \cellcolor{adaptinggreen} 7 & \cellcolor{yellow} 5 & \cellcolor{yellow} 5 & \cellcolor{yellow} 5
		\\ \hline
	\end{tabular}
\end{frame}

\begin{frame}[t]{{\vspace{.3\baselineskip}Quicksort (mit Drei-Wege-Partitionierung)}}
	\textbf{Beispiel} \\
	Partitioniere $A: \KwArray[0..n-1]$ mit Pivotwahl $p(A) := A[\floor{\fract n/3 }]$ {\small (mit $n := \abs{A}$)}
	\\[0,5cm]
	\begin{tabular}{ | c | c | c | c | c | c | c | c | c | }
		\multicolumn{4}{ |c| }{$ < p$}  &\multicolumn{1}{ c| }{ } & \multicolumn{3}{ c| }{ $ = p $} & 
		\\ \hline
		\cellcolor{adaptingred} 3 & \cellcolor{adaptingred} 0 & \cellcolor{adaptingred} 1 & \cellcolor{adaptingred} 1 & \cellcolor{adaptinggreen} 7 &  \cellcolor{yellow} 5 & \cellcolor{yellow} 5 & \cellcolor{yellow} 5 & \cellcolor{adaptinggreen} 7 
		\\ \hline
	\end{tabular}
\end{frame}

\begin{frame}[t]{{\vspace{.3\baselineskip}\hypertarget{label:afterEx2}{}Quicksort (mit Drei-Wege-Partitionierung)}}
	\textbf{Beispiel} \\
	Partitioniere $A: \KwArray[0..n-1]$ mit Pivotwahl $p(A) := A[\floor{\fract n/3 }]$ {\small (mit $n := \abs{A}$)}
	\\[0,5cm]
	\begin{tabular}{ | c | c | c | c | c | c | c | c | c | }
		\multicolumn{4}{ |c| }{$ < p$} & \multicolumn{3}{ c| }{ $ = p $} & \multicolumn{2}{ c| }{ $ > p $}
		\\ \hline
		\cellcolor{adaptingred} 3 & \cellcolor{adaptingred} 0 & \cellcolor{adaptingred} 1 & \cellcolor{adaptingred} 1 & \cellcolor{yellow} 5 & \cellcolor{yellow} 5 & \cellcolor{yellow} 5 & \cellcolor{adaptinggreen} 7 & \cellcolor{adaptinggreen} 7
		\\ \hline
	\end{tabular}
\end{frame}


\begin{frame}{Quicksort (mit Drei-Wege-Partitionierung)}
	Laufzeit, wenn alle Elemente gleich sind? \\
	\pause
	\impl $\Theta(n)$
\end{frame}

\begin{frame}{Quicksort}
	\textbf{Auf Listen} \\
	\begin{itemize}
		\item Wie müsste man vorgehen, um Quicksort auf einfach verketteten Listen anzuwenden (ohne die Liste in ein Array umzuwandeln)? \\ 
		\pause
		\impl Wähle als Pivot $p := head.next$. \\ 
		\textbf{partition}: Laufe durch die Liste und teile Elemente auf zwei Listen $\ell_\leq$ und $\ell_>$ auf. \\
		Sortiere rekursiv $\ell_\leq$ und $\ell_>$ und verbinde sie anschließend.
		%\impl Als $pivot$ das erste Element wählen (das seine Position am Anfang erstmal behält) und ähnlich wie bei Quicksort für Arrays mit Zeigern $p_0$, $p_1$ und $p_2$ die Liste ablaufen ( $(pivot, p_1]$ entspricht $< p$ und $(p_1, p_2]$ entspricht $\geq p$). $p_0$ dient als Vorgänger von $p_1$. Danach wird der $pivot$ mit $p_1$ vertauscht und die rekursiven Aufrufe mit Parameter $(p_1, p_0)$ und $(pivot.next, last)$ durchgeführt.
		\item Wie leicht lässt sich hierbei ein Worst-Case erreichen? Womit könnte man das vermeiden? \\
		\pause
		\impl Wegen eingeschränkter Pivot-Wahl: \\
		Schon fast sortiert \~~> Worst-Case \\
		\impl Viele gleiche Elemente \~~> Drei-Wege-Partition! \\
		\impl Generell: Auf verketteten Listen lieber \textbf{Mergesort}.
		%\impl Das Verfahren leidet unter der eingeschränkten pivot-Wahl und erreicht bei nahezu sortierten Listen und bei Listen mit vielen gleichartigen Elementen ein Worst-Case-ähnliches Verhalten. Dies lässt sich durch das Sortieren mit Mergesort vermeiden.
	\end{itemize}
\end{frame}

\begin{frame}{Quicksort}
	\textbf{...nicht-rekursiv?} \\
	\begin{itemize}
		\item Wie könnte eine iterative Implementierung von Quicksort aussehen? \\
		\pause
		\impl Speichere „Rekursionsparameter“ als Tupel $(\ell, r)$ auf einem \textbf{Stack}, welcher mit einer „großen Schleife“ abgearbeitet wird (\emph{faked recursion})
		\item Was wären mögliche Vorteile/Nachteile? \\ 
		\pause
		\Pros Rekursive Aufrufe werden durch einen platzsparenderen Ersatz gespeichert \\
		\Cons Implementierungsaufwand \impl \textbf{Fehleranfälligkeit}
	\end{itemize}
\end{frame}

\begin{frame}{Quicksort}
	\textbf{... vs. InsertionSort} \\
	\begin{itemize}
		\item Bei „ausreichend kleinen“ Bereichen wird üblicherweise statt einem Rekursionsaufruf \emph{InsertionSort} verwendet. Warum? 
		\pause 
		\implitem \Pros Quicksort gut auf \textbf{größeren} Arrays: Vertauschen einzelner Elemente \textbf{billiger} als ganze Bereiche verschieben \\
		\Cons Quicksort bürokratisch ($O(n^2)$) auf \textbf{kleineren} Arrays: Zu viel Vertauschen $+$ Rekursionsoverhead. 
		\pause
		\implitem \emph{InsertionSort} auf kleinen Arrays linear: „Kurze“ Strecken zum Einsortieren.
		%Quicksort glänzt auf größeren Arrays, da das Vertauschen von Elementen in konstanter Zeit das Verschieben von Bereichen in linearer Zeit ersetzt. Je kleiner das Array, desto eher nähert Quicksort sich $\Theta(n^2)$ an (durch „ineffiziente“ Vertauschungen und Rekursionsoverhead). $InsertionSort$ hingegen hat auf kleinen Arrays lineare Laufzeit, da die Distanz von einem Element zu seiner sortierten Position verhältnismäßig klein ist.
	\end{itemize}
\end{frame}


\begin{frame}[t]{Sortieralgorithmen -- Showdown}
	\vspace{-.6\baselineskip}
	\begin{tabular}{ p{.14\textwidth} | p{.35\textwidth} | p{.4\textwidth} } % c ~ p{5cm}
		& Mergesort & Quicksort \\ 
		\hhline{=|=|=}
		In-place? &\only<2->{\YesCellE{\small Nur auf verketteten Listen*}}& \only<2->{\YesCell*}\\ 
		\hline
		Ablauf & \only<3->{Zuerst Rekursion, danach linearer Aufwand**} & \only<3->{Zuerst linearer Aufwand, danach Rekursion} \\ 
		\hline
		Stabil? & \only<4->{\YesCellE{Möglich}} & \only<4->{\NoCellE{\textbf{Mit Partition}: Nein \newline
				\colorbox{adaptinggreen}{\small (nicht in-place: Möglich)} }} \\ 
		\hline
		Laufzeit & \only<5->{\textbf{garantiert} in $\Theta(n \log n)$} & \only<5->{\textbf{erwartet} in $\Theta(n \log n)$ \newline
			Worst-Case $\Theta(n^2)$ }\\ 
		\hline
		Cache-...  & \only<6->{\NoCellE{unfreundlich}} & \only<6->{\YesCellE{freundlich}} \\ 
		\hline
		& \only<7->{Hat einen sprechenden Namen} & \only<7->{Heißt Quicksort, muss also gut sein} \\
		\hline
	\end{tabular}
	\\[0,1cm]
	\visible<2->{* abgesehen vom Verwaltungsoverhead durch Rekursion \\ } 
	\visible<3->{** abgesehen von Listenzertrennung in linearer Zeit \\ \quad (zur Mitte muss gelaufen werden)}
\end{frame}

\begin{frame}{Bucketsort}
	\textbf{Alles im Eimer?} \\
	\begin{itemize}
		\item $n$ Elemente \textbf{beschränkter} Größe (also $\in \{a, ..., b\} $)
		\pause
		\item Lege an $buckets: \KwArray[a..b] \KwOf \KwListOf \|Element|$ \quad \only<4->{($k := \abs{buckets}$)}
		\item Schmeiße jedes Element $e$ in seinen Eimer: $buckets[e].pushBack(e)$ (hinten anhängen)
		\pause
		\item Am \textbf{Ende}: „Eimer“ zusammenhängen
		\implitem Array sortiert. 
		\pause
		\item \textbf{Laufzeit}: $O(n+k)$
		\item \textbf{Aufpassen} bei großen/unbeschränkten $k$!
	\end{itemize}
\end{frame}

\begin{frame}{Bucketsort}
	\textbf{Generisches Bucketsort} \\
	\begin{itemize}
		\item Eimer nicht unbedingt \textbf{Listen}
		\item Eimer nicht unbedingt nur für \textbf{eine Größe} \\
		\impl \textbf{Intervalle} möglich \\
		\impl In diesem Fall: Buckets müssen am Ende noch \textbf{sortiert} werden! 
		(Z.~B. mit \emph{InsertionSort}) \\
		\impl Dafür empfehlenswert: Elemente \textbf{gleichverteilt} auf Buckets
		\implitem Bucketsort ist \textbf{kein} Sortieralgorithmus für sich, sondern eine \textbf{Familie} von Sortieralgorithmen. 
	\end{itemize}
\end{frame}


\begin{frame}{Bucketsort}
	\underline{Aufgabe:} \\
	Sortiert folgende Liste mit Bucketsort: \\ $\llist{36, 78, 50, 1, 92, 15, 43, 99, 64}$. \\ 
	Verwendet dabei 5 Buckets in den Intervallen: \\
	0 bis 19, 20 bis 39, 40 bis 59, 60 bis 79 und 80 bis 99.
	%Zum Sortieren innerhalb der Eimer wird \emph{InsertionSort} verwendet .
\end{frame}

\begin{frame}{Bucketsort}
	\underline{Lösung:} \\[0,25cm]
	\begin{tabular}{ | c | c | c | c | c | }
		0–19 & 20–39 & 40–59 & 60–79 & 80–99
		\\ \hline
		$\llist{ 1, 15 }$ & $\llist{ 36 }$ & $\llist{ 50, 43 }$ & $\llist{ 78, 64 }$ & $\llist{ 92, 99 }$
		\\ \hline
	\end{tabular}
	\\[0,25cm]
	\impl $\llist{1, 15, 36, 43, 50, 64, 78, 92, 99}$
\end{frame}

\begin{frame}{{}„Nerd's Heaven“{}}
	\pause\pause 
	\textbf{Rumänische Volkstänze FTW!} \\
	\begin{itemize}
		\item InsertionSort: \\
		\url{https://www.youtube.com/watch?v=ROalU379l3U} 
		\item SelectionSort: \\
		\url{https://www.youtube.com/watch?v=Ns4TPTC8whw} 
		\item Mergesort: \\
		\url{https://www.youtube.com/watch?v=XaqR3G\_NVoo} 
		\item Quicksort: \\
		\url{https://www.youtube.com/watch?v=ywWBy6J5gz8} 
	\end{itemize}
	
\end{frame}


\end{document}