% Hier können alte/überholte/verworfene Texte ihre letzte Ruhe finden.
% So geht nichts verloren (Inhalte können wiederauferstehen) und die einzelnen Dateien bleiben trotzdem Übersichtlich.

%-------------------------- 01 --------------------------

\begin{frame}{Algorithmenanalyse}
	\begin{exampleblock}{Ein zauberhafter Algorithmus}
		\begin{algorithm}[H]
			\DontPrintSemicolon
			\footnotesize
			\KwIn{Aufsteigend sortiertes $int$-\KwArray $a$ mit $|a|>0$, Zahl $k \in \Z$}
			\KwOut{$boolean$}
			\Begin {
				$m : int$\;
				$(l, r) := (0, |a|-1)$\;
				\While{$l \neq r$} {
					$m := \lfloor(l+r)/2\rfloor$\;
					\uIf{$k < a[m]$} {
						$r := m$\;
					}
					\uElseIf{$k > a[m]$} {
						$l := m$\;
					}
					\Else {
						\Return{$true$}\;
					}
				}
				\Return{$a[l] = k$}\;
			}
		\end{algorithm}
	\end{exampleblock}
\end{frame}


\begin{frame}{Algorithmenanalyse}
	\begin{itemize}
		\item Es handelt sich um die binäre Suche, die sich die Sortiertheit des Arrays zunutze macht, um echt schneller als in O($n$) zu entscheiden, ob sich ein gesuchtes Element im Array befindet oder nicht
		\pause
		\item Binäre Suche läuft in O($log(n)$), da der Suchbereich bei jedem Schleifendurchlauf halbiert wird
		\pause
	\end{itemize}
	\textbf{Wichtige Aspekte, wenn man die Korrektheit von binärer Suche beweisen will}
	\begin{itemize}
		\pause
		\item Schleifeninvariante: Wenn das gesuchte Element im Array liegt, befindet es sich nicht außerhalb des Suchbereiches. (Am Anfang ist der Suchbereich des gesamte Array)
		\pause
		\item (Nicht immer gefordert:) Zeigen, dass die Schleife terminieren wird, also dass der Suchbereich in jedem Schleifendurchlauf $"$echt kleiner$"$ wird.
	\end{itemize}
\end{frame}

\begin{frame}{Algorithmenanalyse}
	\textbf{Halbwegs formale Beweisskizze} \\[0,125cm]
	\pause
	\begin{itemize}
		\item Invariante: $\forall i \in \{0 ... n-1\} \setminus \{l ... r\}: a[i] \neq k$
		\pause
		\item IA: Invariante gilt offensichtlich $\forall i \in \emptyset$
		\pause
		\item IV: Invariante gilt für vorherigen Ausführungsschritt $(l_0, r_0)$
		\pause
		\item IS: Fallunterscheidung: Ist $k < a[m]$, so sind $a[m] ... a[r_0] > k$, insb. also $a[m] ... a[r_0] \neq k$ (Fall $k > a[m]$ analog).
		\pause
		\item Dass $ | \{l_0 ... m\} | $ (bzw. $ | \{m ... r_0\} | $) $ < | \{l_0 ... r_0\} | $, wird als ersichtlich angenommen
		\pause
		\item Am Ende ist mit $l = r =: s$ $k \neq a[i] \forall i \neq s$ und somit $k \in A \Leftrightarrow a[s] = k$
	\end{itemize}
\end{frame}

\begin{frame}{Algorithmenanalyse}
	\begin{exampleblock}{Ein herausgeforderter Algorithmus}
		\begin{algorithm}[H]
			\DontPrintSemicolon
			\footnotesize
			\KwIn{$int$-\KwArray $a$ mit $|a|>0$, Zahl $k \in \mathbb{Z}$}
			\KwOut{$boolean$}
			\Begin {
				$n := |a|$\;
				\For{$i\ \KwFrom\ 0\ \KwTo\ n-1$} {
					\If{$a[i] = k$} {
						\Return{$true$}\;
					}
				}
				\Return{$false$}\;
			}
		\end{algorithm}
	\end{exampleblock}
	\begin{itemize}
		\item Was macht der Algorithmus?
		\item Laufzeitverhalten?
	\end{itemize}
\end{frame}

\begin{frame}{Algorithmenanalyse}
	\begin{itemize}
		\item Was hier vorliegt, ist die fabulöse lineare Suche, die in linearer Laufzeit $O(n)$ entscheidet, ob ein gesuchtes Element (hier $k$) in einer Datenstruktur enthalten ist.
		\pause
		\item Frage: Wie könnte 
	\end{itemize}
\end{frame}

\begin{frame}{Algorithmenanalyse}
	\textbf{Halbwegs formale Beweisskizze} \\[0,125cm]
	\pause
	\begin{itemize}
		\item Invariante: $\forall j < i: a[i] \neq k$
		\pause
		\item IA: Invariante gilt offensichtlich $\forall i \in \emptyset$
	\end{itemize}
\end{frame}

%-------------------------- 02 --------------------------

\iffalse  %TODO discard?

\begin{frame}{Beispiele Master-Theorem} 
	$n = 8^k, k \in \N_0$: \\[.5\baselineskip]
	$A(n) = $
	\begin{math}
		\begin{cases}
		42,                           & \text{für } n = 1 \\
		8 \cdot A(\frac{n}{8}) + 5n,  & \text{für } n > 1
		\end{cases}
	\end{math} \\[.5\baselineskip]
	\pause
	$\Impl A(n) \in \Theta(n\log n)$
\end{frame}

% add solutions to all


\begin{frame}{Beispiele Master-Theorem}
	$n = 4^k, k \in \N_0$: \\[.5\baselineskip]
	$B(n) = $
	\begin{math}
		\begin{cases}
		1337,                              & \text{für } n = 1 \\
		2 \cdot B(\frac{n}{4}) + 100000n,  & \text{für } n > 1
		\end{cases}
	\end{math} \\[.5\baselineskip]
	\pause
	$\Impl B(n) \in \Theta(n)$
\end{frame}


\begin{frame}{Beispiele Master-Theorem}
	$n = 2^k, k \in \N_0$: \\[0,25cm]
	$C(n) = $
	\begin{math}
		\begin{cases}
		69,                           & \text{für } n = 1 \\
		4 \cdot C(\frac{n}{2}) + 3n,  & \text{für } n > 1
		\end{cases}
	\end{math} \\[0,5cm]
	\pause
	$\Impl C(n) \in \Theta(n^{\log _{2}4}) = \Theta(n^2)$
\end{frame}


\begin{frame}{Beispiele Master-Theorem}
	$n = 13^k, k \in \mathbb{N}_0$: \\[0,25cm]
	$D(n) = $
	\begin{math}
		\begin{cases}
		8,                              & \text{für } n = 1 \\
		11 \cdot D(\frac{n}{13}) + 6n,  & \text{für } n > 1
		\end{cases}
	\end{math} \\[0,5cm]
	\pause
	$\Impl D(n) \in \Theta(n)$
\end{frame}


\begin{frame}{Beispiele Master-Theorem}
	$n = 3^k, k \in \mathbb{N}_0$: \\[0,25cm]
	$E(n) = $ 
	\begin{math}
		\begin{cases}
		255,                            & \text{für } n = 1 \\
		27 \cdot E(\frac{n}{3}) + 3n,   & \text{für } n > 1
		\end{cases}
	\end{math} \\[0,5cm]
	\pause
	$\Impl E(n) \in \Theta(n^{\log _{3}27}) = \Theta(n^3)$
\end{frame}


\begin{frame}{Beispiele Master-Theorem}
	$n = 35767^k, k \in \mathbb{N}_0$: \\[0,25cm]
	$F(n) = $
	\begin{math}
		\begin{cases}
		21,                                   & \text{für } n = 1 \\
		35767 \cdot F(\frac{n}{35767}) + 5n,  & \text{für } n > 1
		\end{cases}
	\end{math} \\[0,5cm]
	\pause
	$\Impl F(n) \in \Theta(n\log n)$
\end{frame}

\fi