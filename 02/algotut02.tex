%beamer

%TODO: Mastertheorem auf Aufgabenfolien dazu! Selsort-Invbsp. doch dranbringen!

% \newboolean{handoutmode}
% \setboolean{handoutmode}{false}
%\newcommand{\handoutmode}{}

%% LaTeX-Beamer template for KIT design
%% by Erik Burger, Christian Hammer
%% title picture by Klaus Krogmann
%%
%% version 2.1
%%
%% mostly compatible to KIT corporate design v2.0
%% http://intranet.kit.edu/gestaltungsrichtlinien.php
%%
%% Problems, bugs and comments to
%% burger@kit.edu
\ifdefined \handoutmode
\documentclass[18pt, handout]{beamer}
\else
\documentclass[18pt]{beamer}
\fi

\usepackage[T1]{fontenc}
\usepackage[utf8]{inputenc}

\usepackage{../preamble/templates/beamerthemekit}

\usepackage[vlined]{algorithm2e}  %possible: noend, noline, ...
\usepackage{amssymb}
\usepackage{amsmath}
\usepackage{wasysym}
\usepackage{graphicx}
%\usepackage{hyperref}
\usepackage[export]{adjustbox}
\usepackage{wrapfig}
\usepackage{colortbl}
\usepackage{tikz}
\usetikzlibrary{matrix}
\usetikzlibrary{arrows.meta}
\usetikzlibrary{automata}
\usetikzlibrary{tikzmark}
\graphicspath{{images/}}
%\usepackage[colorlinks=true,urlcolor=blue,linkcolor=blue]{hyperref}
\usepackage[outline]{contour}
\usepackage{cancel}
\usepackage[warn]{textcomp}
\usepackage{multicol}
\usepackage{tabularx}
\usepackage{xcolor}
\usepackage{hhline}
\usepackage{environ}
\usepackage{calc}
\usepackage{bm}
\usepackage{xspace} % for \xspace command
\usepackage{varwidth}
\usepackage{csquotes}

\newcommand{\mycomment}[1]{}

%%%% CONFIG

\input{../preamble/config.tex}

%%%% CONFIG END

%\renewcommand{\SS}{\iffontchar\font"1E9E \symbol{"1E9E}\else SS\fi} % SHAME ON YOU, LATEX!
\newcommand{\TM}{\text{$\mbox{}^\text{\tiny TM}$}}
\newcommand{\pluseq}{\mathrel{+}=}
\newcommand{\pp}{\operatorname{++}} 
\newcommand{\mm}{\operatorname{--\mbox{\:}--}}
\newcommand{\minuseq}{\mathrel{-}=}
\newcommand{\asteq}{\mathrel{*}=}
\newcommand{\muleq}{\asteq}
\renewcommand{\mod}{\mathop{\textbf{mod}}} 
\renewcommand{\div}{\mathop{\textbf{div}}}
\newcommand{\N}{\mathbb{N}} 
\newcommand{\R}{\mathbb{R}}
\newcommand{\Z}{\mathbb{Z}}
\newcommand{\E}{\mathbb{E}}
\renewcommand{\P}{\mathbb{P}}
\newcommand{\BB}{\mathbb{B}} % \B already exists
\newcommand{\NP}{\ensuremath{\mathcal{N\hspace{-1.5pt}P}}}
\newcommand{\Oh}[1]{\mathcal{O}\!\left(#1\right)}
\renewcommand{\O}{\mathcal{O}}
\newcommand{\Om}[1]{\Omega\!\left(#1\right)}
\newcommand{\Th}[1]{\Theta\!\left(#1\right)}

\newcommand{\realTilde}{\textasciitilde\xspace}
\renewcommand{\qedsymbol}{\textcolor{black}{\openbox}}

\newcommand{\size}[1]{\ensuremath{\left\lvert #1 \right\rvert}}
\newcommand{\set}[1]{\left\{#1\right\}}
\newcommand{\tuple}[1]{\left(#1\right)}

\newcommand*{\from}{\colon}

\newcommand{\morescalingdelimiters}{   % for proper \left( \right) typography
	\delimitershortfall=0pt  % formerly: 0pt  
	\delimiterfactor=1
}
% todo later
%\delimitershortfall=0pt  % for proper \left( \right) typography
%\delimiterfactor=1

% --- \frameheight constant ---
\newlength\fullframeheight
\newlength\framewithtitleheight
\setlength\fullframeheight{.92\textheight}
\setlength\framewithtitleheight{.86\textheight}

\newlength\frameheight
\setlength\frameheight{\fullframeheight}

\let\frametitleentry\relax
\let\oldframetitle\frametitle
\def\frametitle#1{\global\def\frametitleentry{#1}\if\relax\frametitleentry\relax\else\setlength\frameheight{\framewithtitleheight}\fi\oldframetitle{#1}}

% --- \frameheight constant end ---

\def\·{\cdot}
\def\*{\cdot}
\def\<{\langle}
\def\>{\rangle}


\newcommand{\zB}{z.\,B.\@\xspace}
\newcommand{\ZB}{Z.\,B.\@\xspace}

\newcommand{\ceil}[1]{\left\lceil#1\right\rceil}
\newcommand{\floor}[1]{\left\lfloor#1\right\rfloor}
\newcommand{\abs}[1]{\left|#1\right|}
\newcommand{\Matrix}[1]{\begin{pmatrix} #1 \end{pmatrix}}
\newcommand{\braced}[1]{\left\lbrace #1 \right\rbrace}
\newcommand{\llist}[1]{\langle #1 \rangle}
\newcommand{\Mid}{\;\middle|\;}

\let\after\circ

\newcommand{\entspr}{\ensuremath{\mathrel{\hat{=}}}\xspace}

\def\~~>{\ensuremath{\rightsquigarrow}}  % FuCKING FINALLY! :D

% "something" placeholder. Useful for repairing spacing of operator sections, like `\sth = 42`.
\def\sth{\vphantom{.}}

\def\fract#1/#2 {\frac{#1}{#2}}  % ! TRAILING SPACE is CRUCIAL!
\def\dfract#1/#2 {\dfrac{#1}{#2}} % ! Trailing space is crucial!

\newcommand{\tight}[1]{{\renewcommand{\arraystretch}{0.76} #1}}
\newcommand{\stackedtight}[1]{{\renewcommand{\arraystretch}{0.76} \begin{matrix} #1 \end{matrix}} }
\newcommand{\stacked}[1]{\begin{matrix} #1 \end{matrix} }
\newcommand{\casesl}[1]{\delimitershortfall=0pt  \left\lbrace\hspace{-.3\baselineskip}\begin{array}{ll} #1 \end{array}\right.}
\newcommand{\casesr}[1]{\delimitershortfall=0pt  \left.\begin{array}{ll} #1 \end{array}\right\rbrace}
\newcommand{\caseslr}[1]{\delimitershortfall=0pt  \left\lbrace\begin{array}{ll} #1 \end{array}\hspace{-.3\baselineskip}\right\rbrace}

\def\q#1uad{\ifnum#1=0\relax\else\quad\q{\the\numexpr#1-1\relax}uad\fi}
% e.g. \q1uad = \quad, \q2uad = \qquad etc.

\newcommand{\qqquad}{\q3uad}


\def\indentstring{}
\def\§#1{\def\indentstring{#1}#1}
\def\.{{$\hphantom{\text{\indentstring}}$}}


\newcommand{\impl}{\ifmmode\ensuremath{\mskip\thinmuskip\Rightarrow\mskip\thinmuskip}\else$\Rightarrow$\xspace\fi}  
\newcommand{\Impl}{\ifmmode\implies\else$\Longrightarrow$\xspace\fi}

\newcommand{\gdw}{\ifmmode\mskip\thickmuskip\Leftrightarrow\mskip\thickmuskip\else$\Leftrightarrow$\xspace\fi}
\newcommand{\Gdw}{\ifmmode\iff\else$\Longleftrightarrow$\xspace\fi}

\newcommand{\symbitemnegoffset}{\hspace{-.33\baselineskip}}
\newcommand{\implitem}{\item[\impl\symbitemnegoffset]}
\newcommand{\Implitem}{\item[\Impl\symbitemnegoffset]}


\newcommand{\forcenewline}{\mbox{}\\}

\newcommand{\bfalert}[1]{\textbf{\alert{#1}}}
\let\elem\in   % I'm a Haskell freak. Don't judge me. :P


\newenvironment{threealign}{%
	\[
	\begin{array}{r@{\ }c@{\ }l}
}{%
	\end{array}	
	\]
}


\makeatletter
% Provides color if undefined.
\newcommand{\colorprovide}[2]{%
	\@ifundefinedcolor{#1}{\colorlet{#1}{#2}}{}}
\makeatother



%\pgfdeclarelayer{background}
%\pgfdeclarelayer{foreground}
%\pgfsetlayers{background,main,foreground}

\colorprovide{lightred}{red!30}
\colorprovide{lightgreen}{green!40}
\colorprovide{lightyellow}{yellow!50}
\colorprovide{beamerlightred}{lightred}
\colorprovide{beamerlightgreen}{lightgreen}
\colorprovide{beamerlightyellow}{lightyellow}
\colorprovide{fullred}{red!60}
\colorprovide{fullgreen}{green}
\definecolor{darkred}{RGB}{115,48,38}
\definecolor{darkgreen}{RGB}{48,115,38}
\definecolor{darkyellow}{RGB}{100,100,0}

\only<handout:0>{\colorlet{adaptinglightred}{beamerlightred}}
\only<handout:0>{\colorlet{adaptinglightgreen}{beamerlightgreen}}
\only<handout:0>{\colorlet{adaptinglightyellow}{beamerlightyellow}}
\only<beamer:0>{\colorlet{adaptinglightred}{lightred}}
\only<beamer:0>{\colorlet{adaptinglightgreen}{lightgreen}}
\only<beamer:0>{\colorlet{adaptinglightyellow}{lightyellow}}
\only<handout:0>{\colorlet{adaptingred}{lightred}}
\only<beamer:0>{\colorlet{adaptingred}{fullred}}
\only<handout:0>{\colorlet{adaptinggreen}{lightgreen}}
\only<beamer:0>{\colorlet{adaptinggreen}{fullgreen}}

\colorlet{checkgreen}{green!80}
\colorlet{crashred}{fullred}
\colorprovide{myalertcolor}{red}
\colorlet{alertcolor}{myalertcolor}

\definecolor{kwblue}{rgb}{0.3,0.3,1}
\definecolor{strcolor}{RGB}{48,115,38}

\newcommand{\str}[1]{\shorthandoff{"}\textcolor{strcolor}{\text{"{}#1"{}}\shorthandon{"}}}

\newcommand{\gray}[1]{\textcolor{gray}{#1}}

\newcommand{\MyKwSty}[1]{\textcolor{kwblue}{\textbf{#1}}}
\SetKwSty{MyKwSty}

\SetArgSty{textnormal} % to end conditional italics madness

\newcommand{\MyCommentSty}[1]{\emph{\gray{#1}}}
\SetCommentSty{MyCommentSty}

\SetKwComment{Comment}{// }{}

\newcommand{\LComment}[1]{\Comment*[h]{#1}}
\newcommand{\RComment}[1]{\quad \Comment*[h]{#1}}



\SetKwBlock{KwFunc}{function}{}
\SetKwBlock{KwProc}{procedure}{}
\newcommand{\Function}[2]{\KwFunc({#1}){#2}}
\newcommand{\Procedure}[2]{\KwProc({#1}){#2}}
\SetKwBlock{KwEmptyBlock}{}{}
\newcommand{\EmptyBlock}[1]{\KwEmptyBlock(){#1}}

% Binary operator keywords (small surrounding spaces)
\newcommand{\SetKwBin}[2]{
	\expandafter\newcommand\csname #1\endcsname{\ensuremath{\mathbin{\KwSty{#2}}}}	
}
% Relational operator keywords (bigger surrounding spaces)
\newcommand{\SetKwRel}[2]{
	\expandafter\newcommand\csname #1\endcsname{\ensuremath{\mathrel{\KwSty{#2}}}}	
}
% Directive keywords (trailing space)
\newcommand{\SetKwDir}[2]{
	\expandafter\newcommand\csname #1\endcsname{\ensuremath{\mathop{\KwSty{#2}}}}		
}

\DontPrintSemicolon
%\SetKwSwitch{Switch}{Case}{Other}{switch on}{}{}{else}{}{}

%\newcommand{\SwitchCase}[2]{\KwSty{case} #1 \KwOf\EmptyBlock{#2}}
%\newcommand{\case}[2]{#1:\EmptyBlock{#2}}
\SetKwDir{KwAssert}{assert}
\SetKwDir{KwInvariant}{invariant}
\SetKwRel{KwStep}{step}
\SetKwRel{KwDownto}{downto}	
\SetKwDir{KwArrayOf}{array of\,}
\SetKwDir{KwArray}{array}
\let\KwTo\undefined
\SetKwRel{KwTo}{to}
\SetKwRel{KwOf}{of}
\let\KwInput\KwIn
\let\KwIn\undefined
\SetKwRel{KwIn}{in}
\SetKwRel{KwInto}{into}
\SetKwDir{KwNot}{not}
\SetKwRel{KwIs}{is}
\SetKwRel{KwAnd}{and}
\SetKwRel{KwOr}{or}
\SetKwBin{KwMod}{mod}
\SetKwBin{KwDiv}{div}
\SetKwDir{KwContinue}{continue}
\SetKwDir{KwBreak}{break}
\SetKwDir{KwThrow}{throw}
\SetKw{KwTrue}{true}
\SetKw{KwFalse}{false}
\SetKw{KwThis}{this}
\SetKwDir{KwNew}{new}
\SetKwRel{KwFrom}{from}
\SetKwDir{KwFor}{for}
\SetKwDir{KwEach}{each}
\SetKw{KwProcedure}{procedure}
\SetKw{KwMethod}{method}
\SetKw{KwFunction}{function}
\SetKwDir{KwPointerTo}{Pointer to}
\SetKwData{KwList}{List}
\SetKwData{KwSet}{Set}
\newcommand{\Element}{\|Element|}
\newcommand{\KwListOf}{\ensuremath{\mathop{\KwList \KwOf}}} 
\newcommand{\KwSetOf}{\ensuremath{\mathop{\KwSet \KwOf}}} 
\SetKwDir{KwDispose}{dispose}


\def\|#1|{\text{\normalfont #1}}  % | steht für senkrecht (anstatt kursiv wie sonst im math mode)

% proper math typography
\newcommand{\functionto}{\longrightarrow} 
\renewcommand{\geq}{\geqslant}
\renewcommand{\leq}{\leqslant}
\let\oldsubset\subset
\renewcommand{\subset}{\subseteq} % for all idiots out there using subset

\newcommand{\access}{\text{\textrightarrow}} 
\def\->{\access}

\let\oldemptyset\emptyset
\let\emptyset\varnothing % proper emptyset

\newcommand{\stdarraystretch}{1.20}
\renewcommand{\arraystretch}{\stdarraystretch}  % for proper row spacing in tables

\newcommand{\mailto}[1]{\href{mailto:#1}{{\textcolor{blue}{\underline{#1}}}}}
\newcommand{\urlnamed}[2]{\href{#1}{\textcolor{blue}{\underline{#2}}}}
\renewcommand{\url}[1]{\urlnamed{#1}{#1}}

\newcommand{\hanging}{\hangindent=0.7cm}
\newcommand{\indented}{\hanging}

\newcommand{\Pros}{{\huge \protect\textcolor{adaptinggreen}{\protect\contour{black}{\raisebox{-.3pt}{$\protect\textbf{+}$}}}}\xspace}

\newcommand{\Cons}{\hspace{1pt}\protect\scalebox{0.88}[1]{\huge \protect\contour{black}{\protect\textcolor{adaptingred}{\raisebox{-1pt}{$\protect\textbf{--}$}}}}\hspace{1pt}\xspace}

\newcommand{\yop}{\textcolor{checkgreen}{\protect\contour{black}{\protect\textbf{\checked}}}\xspace}
\newcommand{\crash}{\ensuremath{\textcolor{crashred}{\protect\contour{black}{\protect\textbf{\lightning}}}}\xspace}

\newcommand{\YesCellE}[1]{\cellcolor{adaptinggreen} {#1}}
\newcommand{\YesCell}{\YesCellE{\textbf{Ja}}}
\newcommand{\NoCellE}[1]{\cellcolor{adaptingred} {#1}}
\newcommand{\NoCell}{\NoCellE{\textbf{Nein}}}


\newcommand{\TrueQuestion}[1]{
	\TrueQuestionE{#1}{}
}

\newcommand{\YesQuestion}[1]{
	\YesQuestionE{#1}{}
}

\newcommand{\FalseQuestion}[1]{
	\FalseQuestionE{#1}{}
}

\newcommand{\NoQuestion}[1]{
	\NoQuestionE{#1}{}
}

\newcommand{\DependsQuestion}[1]{
	\DependsQuestionE{#1}{}
}

\newcommand{\QuestionVspace}{\vspace{4pt}}
\newcommand{\QuestionParbox}[1]{\begin{varwidth}{.85\linewidth}#1\end{varwidth}}
\newcommand{\ExplanationParbox}[1]{\begin{varwidth}{.99\linewidth}#1\end{varwidth}}
\colorlet{questionlightgray}{gray!23}
\let\defaultfboxrule\fboxrule

% #1: bg color
% #2: fg color short answer
% #3: short answer text
% #4: question
% #5: explanation
\newcommand{\GenericQuestion}[5]{
	\setlength\fboxrule{2pt}
	\only<+|handout:0>{\hspace{-2pt}\fcolorbox{white}{questionlightgray}{\QuestionParbox{#4} \quad\textbf{?}}}
	\visible<+->{\hspace{-2pt}\fcolorbox{white}{#1}{\QuestionParbox{#4} \quad\textbf{\textcolor{#2}{#3}}} \ExplanationParbox{#5}} \\
	\setlength\fboxrule{\defaultfboxrule}
}

% #1: Q text
% #2: Explanation
\newcommand{\TrueQuestionE}[2]{
	\GenericQuestion{adaptinglightgreen}{darkgreen}{Wahr.}{#1}{#2}
}

% #1: Q text
% #2: Explanation
\newcommand{\YesQuestionE}[2]{
	\GenericQuestion{adaptinglightgreen}{darkgreen}{Ja.}{#1}{#2}
}

% #1: Q text
% #2: Explanation
\newcommand{\FalseQuestionE}[2]{
	\GenericQuestion{adaptinglightred}{darkred}{Falsch.}{#1}{#2}
}

% #1: Q text
% #2: Explanation
\newcommand{\NoQuestionE}[2]{
	\GenericQuestion{adaptinglightred}{darkred}{Nein.}{#1}{#2}
}

% #1: Q text
% #2: Explanation
\newcommand{\DependsQuestionE}[2]{
	\GenericQuestion{adaptinglightyellow}{darkyellow}{Je nachdem!}{#1}{#2}
}

\newenvironment{headframe}{\Huge THIS IS AN ERROR. PLEASE CONTACT THE ADMIN OF THIS TEX CODE. (headframe env def failed)}{}
\RenewEnviron{headframe}[1][]{
	\begin{frame}\frametitle{\ }
		\centering 
		\Huge\textbf{\textsc{\BODY} \\
		} 
		\Large {#1}
		\frametitle{\ }
	\end{frame}
}

\newcommand{\sectionheadframe}[2]{
	\section{#1}
	\begin{headframe}[#2]
		#1
	\end{headframe}	
}

\newcommand{\slideThanks}{
	\begin{frame}{Credits}
		%\begin{block}{}
			Vorgänger dieses Foliensatzes wurden erstellt von: \\[1em]
			Christopher Hommel  (urspr. Verfasser)\\
			Daniel Jungkind 
		%\end{block}
	\end{frame}
}

%% SLIDE FORMAT

% use 'beamerthemekit' for standard 4:3 ratio
% for widescreen slides (16:9), use 'beamerthemekitwide'


% \usepackage{../preamble/templates/beamerthemekitwide}

%% TITLE PICTURE

% if a custom picture is to be used on the title page, copy it into the 'logos'
% directory, in the line below, replace 'mypicture' with the 
% filename (without extension) and uncomment the following line
% (picture proportions: 63 : 20 for standard, 169 : 40 for wide
% *.eps format if you use latex+dvips+ps2pdf, 
% *.jpg/*.png/*.pdf if you use pdflatex)
\IfFileExists{images/logo.png}{
	\titleimage{logo}
}{}
\IfFileExists{images/logo.jpg}{
	\titleimage{logo}
}{}

%% TITLE LOGO

% for a custom logo on the front page, copy your file into the 'logos'
% directory, insert the filename in the line below and uncomment it

\titlelogo{empty}

% (*.eps format if you use latex+dvips+ps2pdf,
% *.jpg/*.png/*.pdf if you use pdflatex)

%% TikZ INTEGRATION

% use these packages for PCM symbols and UML classes
% \usepackage{templates/tikzkit}
% \usepackage{templates/tikzuml}

% the presentation starts here


%% Titel einfügen
\newcommand{\titleframe}{\frame{\titlepage}}

\newcounter{weeknum}

\newcounter{tasknum}
\newcounter{subtasknum}
\resetcounteronoverlays{subtasknum}
\resetcounteronoverlays{tasknum}
\let\oldthesubtasknum\thesubtasknum
\def\thesubtasknum{\ifnum\oldthesubtasknum=0\relax\else\alph{subtasknum})\fi}
\def\ThisHasSubtasks{\setcounter{subtasknum}{1337}}
\def\thetasknumminusone{\the\numexpr\thetasknum-1\relax\xspace}
\newcommand{\taskheading}[1]{\ifnum\oldthesubtasknum=1337\relax\setcounter{subtasknum}{1}\else\setcounter{subtasknum}{0}\fi\addtocounter{tasknum}{1}\textbf{Aufgabe \thetasknum\thesubtasknum: #1} \\}
\newcommand{\subtaskheading}[1]{\addtocounter{subtasknum}{1}\textbf{Aufgabe \thetasknum\thesubtasknum: #1} \\}
\newcommand{\solutionheading}{\textbf{Lösung zu Aufgabe \thetasknum\thesubtasknum} \\}

\setbeamertemplate{section in toc}{
	\gray{\inserttocsection} \par	
}
\setbeamertemplate{navigation symbols}{}

\newif\ifprinttableofcontents \printtableofcontentstrue
\def\notableofcontents{\printtableofcontentsfalse}
\let\notoc\notableofcontents

%% Alles starten mit \starttut{X}
\newcommand{\starttut}[1]{\setcounter{weeknum}{#1}\pdfinfo{
		/Author (\myname)
		/Title  (Algorithmen-Tutorium \mytutnumber, Woche \theweeknum)
	}\titleframe
	\ifprinttableofcontents\frame{\frametitle{Inhalt}\tableofcontents}\fi
	\mycomment{
		\AtBeginSection[]{%
			\begin{frame}{Wo sind wir gerade?}
				\tableofcontents[currentsection]
			\end{frame}\addtocounter{framenumber}{-1}
		}
	}	
}


\newcommand{\framePrevEpisode}{
	\begin{headframe}
		\mylasttimestext
	\end{headframe}
}

\newcommand{\lastframetitled}[6]{
	\frame{\frametitle{#6}
		\vspace{-#2\baselineskip}
		\begin{figure}[H]
			\centering
			\LARGE \textbf{\textsc{#5}} \\
			\vspace{.2\baselineskip}
			\includegraphics[#1]{#3}
			\vspace{-10pt}
			\begin{center}
				\small \url{#4} 
			\end{center}
		\end{figure} 
	}
}

% #1 number
% #2 title 
% #3 vspace (positive) without unit (\baselineskip)
\newcommand{\xkcdframe}[3]{
	\lastframetitled{width=.96\textwidth}{#3}{xkcd_#1}{http://xkcd.com/#1}{}{#2}
}

\newcommand{\xkcdframevert}[3]
{
	\lastframetitled{height=.96\frameheight}{#3}{xkcd_#1}{http://xkcd.com/#1}{}{#2}
}

\newif\ifisWS \isWSfalse

\def\semesterWS{\isWStrue}
\def\semesterSS{\isWSfalse}

\semesterSS

\def\semesterstring{\ifisWS WS \thisyear/\the\numexpr\nextyear-2000\relax\else SS \thisyear\fi}

\edef\nextyear{\the\numexpr\thisyear+1\relax} 

\title[Algorithmen-Tutorium \mytutnumber, Woche \theweeknum]{Algorithmen I \\[-2pt] Tutorium \mytutnumber}
\subtitle{Woche \theweeknum\ |\xspace\mydate{\theweeknum}}


\author[\myname]{{\mynamebold \; (\mailto{\mymail})}}

\institute{Institut für Theoretische Informatik}

\date{\mydate{\theweeknum}\ }



% Bibliography
% not needed here:
%\usepackage[citestyle=authoryear,bibstyle=numeric,hyperref,backend=biber]{biblatex}
%\addbibresource{templates/example.bib}
%\bibhang1em

% presentation

\setbeamercovered{transparent=1}  %min=0, max=100

% change the following line to "ngerman" for German style date and logos
\selectlanguage{ngerman}

\ifnum\thisyear=2018 \else \errmessage{Old ILIAS link inside preamble. Please update.} \fi

\newcommand{\ILIAS}{\urlnamed{https://ilias.studium.kit.edu/ilias.php?ref_id=808428&cmdClass=ilrepositorygui&cmdNode=k8&baseClass=ilrepositorygui}{ILIAS}\xspace} 

\newcommand{\Socrative}{\only<handout:0>{socrative.com $\qquad$ \~~> Student login \\ Raumname:  \mysocrativeroom\\ \medskip}}

\newcommand{\thasse}[1]{
	\ifdefined\ThassesTut #1\xspace \else\fi
}
\newcommand{\daniel}[1]{
	\ifdefined\DanielsTut #1\xspace \else\fi
}
\newcommand{\thassedaniel}[2]{\ifdefined\ThassesTut #1\else\ifdefined\DanielsTut #2\fi\fi\xspace}

\ifdefined\ThassesTut \ifdefined\DanielsTut \errmessage{ERROR: Both ThassesTut and DanielsTut flags are set. This is most likely an error. Please check your config.tex file.} \else \fi \else \ifdefined\DanielsTut \else \errmessage{ERROR: Neither ThassesTut  nor DanielsTut flags are set. This is most likely an error. Please check your config.tex file.} \fi\fi

\begin{document}

\starttut{2}

% presentation

\iffalse

\begin{frame}{Zum letzten Übungsblatt (Nr. 1)}
	\textbf{Aufgabe 4} \\[0,125cm]
	\begin{itemize}
		\item Man hat ein Array $A$ von (beliebigen) Elementen und möchte zu jedem Element $"$Zusatzinformationen$"$ anlegen (hier: Ein $bool(ean)$ pro Zahl, ob diese zusammengesetzt ist oder nicht)
		\item Wenn man beim Abrufen der Zusatzinformation eines Elementes $e$ auch den Index $i$ von $e$ in $A$ hat (wie es hier der Fall ist, da über den Inhalt von $A$ iteriert wird), bietet es sich an, diese Zusatzinformationen in einem Array $B$ der Größe $|A|$ zu speichern, also $Zusatzinformation(A[i]) = B[i]$
	\end{itemize}
\end{frame}

\begin{frame}{Zum letzten Übungsblatt (Nr. 1)}
	\textbf{Aufgabe 4} \\[0,125cm]
	\begin{itemize}
		\item In Bezug auf die Aufgabe im Übungsblatt also: \\
		\footnotesize
		\begin{exampleblock}{ }
			\begin{algorithm}[H]
				\DontPrintSemicolon
				$function\ decideIfZusammengesetzt(M : Sorted\ Array\ of \Z) : Array\ of\ bool(ean)$\;
				\Begin{
					$B[|M|] : Array\ of\ bool(ean)$\;
					\For{$i\ \KwFrom\ 0\ \KwTo\ |M|-1$} {
						//Entscheide, ob $M[i]$ zusammengesetzt ist\;
						\uIf{$M[i]\ ist\ zusammengesetzt$} {
							$B[i] := true$\;
						}
						\Else {
							$B[i] := false$\;	
						}
					}
					\KwRet{$B$}\;
				}
			\end{algorithm}
		\end{exampleblock}
	\end{itemize}
\end{frame}


\begin{frame}{Zum letzten Übungsblatt (Nr. 1)}
	\textbf{O-Kalkül} \\[0,25cm]
	\begin{itemize}
		\item Aufgabe 1 war sehr mathematisch, dafür können die bewiesenen Zusammenhänge künftig als gegeben betrachtet werden (falls die Aufgabenstellung keinen erneuten Beweis verlangt)
		\item Zur Angabe von Laufzeiten
		\begin{itemize}
			\item Falls eine obere Schranke gefordert wird $\Rightarrow O(f(n))$ (potenziell $"$zu große$"$ Schranke) ausreichend
			\item Falls eine scharfe asymptotische Grenze gefordert wird $\Rightarrow \Theta(f(n))$ benötigt
			\item Falls die Laufzeit eines (spezifischen) Algorithmus angegeben bzw. bestimmt werden soll $\Rightarrow \Theta(f(n))$ gefordert
		\end{itemize}
	\end{itemize}
\end{frame}

\fi

\begin{frame}{Zum letzten Blatt (\#1)} %TODO adapt to current sheet
	\begin{itemize}
		\item \textbf{Aussage} + \textbf{Begründung} irgendwie deutlich machen
		\item \textbf{Form}: Definieren, was ihr benutzt (Wo kommen $f$ und $g$ her??)
		\pause
		\item Induktion: Hübsche Rechnung. \textbf{Und was heißt das jetzt?} (Zusammenhang Rechnung $\Leftrightarrow$ Code-Geschehen nicht vergessen!)
		\item Induktion mittels $a \rightsquigarrow a + 1$ gefährlich: Wo wird a erhöht? \\ ($c$ genauso: Wird in der \textit{Mitte} erhöht!) \\
		$\impl$ Besser: Schleifendurchläufe \textbf{nummerieren} ($i$) und Variablen auch ($a_i, b_i,...$)!
	\end{itemize} 
\end{frame}

\begin{frame}{Zum letzten Blatt (\#1)}
	\begin{exampleblock}{Aufgabe 2 c)}
		\begin{algorithm}[H]
			\Function{f$(n, m : \N): (\N_0, \N_0)$}{
				$a = 0 : \N_0$ \;
				$b = m : \N_0$ \;
				$c = 1 : \N_0$ \;
				\While{$m - c \cdot n \geq 0$}{
					\only<2>{$\KwInvariant m = a \cdot n + b$ \;}
					$a := c$ \;
					$c := c+1$ \;
					$b := m - a \cdot n$ \;
				}
				\Return{$(a,b)$}
			}
		\end{algorithm}
	\end{exampleblock}
\end{frame}

\begin{frame}{Lösung Aufgabe 2 c)}
	Bezeichne $i$ die Nummer des aktuellen Schleifendurchlaufs und $a_i, b_i, c_i$ den Wert von a, b, c zu Beginn von Schleifendurchlauf $i$. \\
	\pause
	\hanging \textbf{IA}. ($i=1$): \quad $a_1 \cdot n + b_1 = 0 \cdot n + m = m. \quad \checkmark$ \\
	\pause
	\hanging \textbf{IV}.: Die Invariante galt zu Beginn von Schleifendurchlauf $i$. \\
	\pause
	\hanging {\textbf{IS}. ($i \rightsquigarrow i + 1$): Es gilt zu Beginn von Schleifendurchlauf $i+1$: \newline
		$a_{i+1} = c_i$, \newline
		$b_{i+1} = m - a_{i+1} \cdot n$. \newline
		\hspace*{-0.4cm}$\Impl a_{i+1} \cdot n + b_{i+1} = c_i \cdot n + \left(m - c_i \cdot n\right) = m. \qed$
	}
\end{frame}

\sectionheadframe{Laufzeiten}{...denn Zeit ist Geld}

\begin{frame}{Laufzeitabschätzung}
	\vspace{-1.5\baselineskip}
	\begin{columns}[T] 
		\begin{column}[T]{.45\textwidth} 
			\begin{exampleblock}{Laufzeit?}
				\begin{algorithm}[H]
					\Function{boing$(n: \N): \N$}{
						$k := 0$\;
						$\ell := 0$\;
						\For{$i := 1 \KwTo n$} {
							$\ell\pp$\;
							\If{$i > n - 4$} {
								\For{$j := 1 \KwTo n$} {
									$k\pp$\;
								}
							}
						}
						\KwRet{$k+\ell$}\;
					}
				\end{algorithm}
			\end{exampleblock}
		\end{column}
		\pause
		\begin{column}[T]{.52\textwidth} 
			\begin{itemize}
				\item Erster Gedanke: \\ 
				\textbf{Äußere} Schleife: $n$ Durchläufe, \\ 
				\textbf{Innere} Schleife: $n$ Durchläufe \\ 
				\only<2|handout:0>{\impl Laufzeit in $\Theta(n \cdot n) = \Theta(n^2)$}
				\only<3>{\impl Laufzeit in $\xcancel{\Theta(n \cdot n) = \Theta(n^2)}$}
				\pause
				\item \textbf{Aber}: Innere Schleife wird nur \textbf{max. 4x} erreicht \\ (nämlich für \\  $i \in \{n-3, n-2, n-1, n\}$) \\
				\medskip
				$\impl \text{Laufzeit in } \Theta(n + 4n) = \Theta(n)$
			\end{itemize}
		\end{column}
	\end{columns}
\end{frame}


\begin{frame}{Laufzeitabschätzung}
	\begin{exampleblock}{Laufzeit?}
		\begin{algorithm}[H]
			\Function{doing$(n: \N): \N$}{
				$k := 0$\;
				$\ell := 0$\;
				\For{$i := 1\ \KwTo\ n$} {
					$\ell\pp$\;
					\For{$j := i\ \KwTo\ n$} {
						$k\pp$\;
					}
				}
				\KwRet{$k+\ell$}\;
			}
		\end{algorithm}
	\end{exampleblock}
\end{frame}

\begin{frame}{Laufzeitabschätzung}
	\begin{itemize}
		\item Erster Gedanke: Äußere Schleife macht $n$-mal „irgendwas“ $\impl n \cdot (???)$ \\
		(Klammer? Wie schreiben wir das auf? Nicht $n$-mal dasselbe, sondern \textbf{von $i$ abhängig}!) 
		\pause
		\item Rettung: Anzahl innere Schleifendurchläufe \textbf{einzeln} für jedes $i = 1,\dots,n$ \textbf{aufsummieren}!
		\pause
		\item Für ein festes $i$ wird innere Schleife $(n-i+1)$-mal durchlaufen
		\pause
		\impl Gesamtanzahl der inneren Schleifendurchläufe:
		\begin{align*}
			&\sum\limits_{i=1}^n (n-i+1) =
			\sum\limits_{i=1}^n n - \sum\limits_{i=1}^n i + \sum\limits_{i=1}^n 1 = 
			n^2 - \sum\limits_{i=1}^n i + n      \\ \pause
			\hphantom{.} = \hphantom{.}&n^2 - \frac{n \cdot (n+1)}{2} + n =
			n^2 - \frac{n^2}{2} + n - \frac{n}{2} =
			\frac{n^2 + n}{2} \in \Theta(n^2).
		\end{align*}
	\end{itemize}
\end{frame}


\begin{frame}{Laufzeitabschätzung}
	\begin{columns}[T] 
		\begin{column}[T]{.45\textwidth}
			\vspace{-\baselineskip} 
			\begin{exampleblock}{Laufzeit?}
				\begin{algorithm}[H]
					\Function{going$(n: \N): \N$}{
						$k := 0$\;
						$\ell := 0$\;
						\For{$i := 1 \KwTo n$} {
							$\ell\pp$\;
							\If{$i > n - 4$} {
								\For{$j := i \KwTo n$} {
									$k\pp$\;
								}
							}
						}
						\KwRet{$k+\ell$}\;
					}
				\end{algorithm}
			\end{exampleblock}
		\end{column}
		\pause
		\begin{column}[T]{.53\textwidth} 
			Die innere Schleife wird wieder \textbf{nur max. vier Mal} erreicht (s. \emph{boing})  \\ 
			(nämlich für $i \in \{n-3, n-2, n-1, n\}$). \\ 
			\smallskip \pause
			\impl Die innere Schleife wird erst vier-, dann drei-, dann zwei- und dann einmal durchlaufen \\ 
			\medskip
			\impl Laufzeit in $\Theta(n + 4+3+2+1) = \Theta(n)$
		\end{column}
	\end{columns}
	
\end{frame}

\begin{frame}{Laufzeitabschätzung}
	\textbf{Hinweise für Aufgaben} \\[0,25cm]
	\begin{itemize}
		\pause
		\item „Obere Schranke“ gefordert \\ 
		$\impl O(f(n))$ (potenziell „zu große“ Schranke) ausreichend
		\pause
		\item „\textbf{Scharfe} asymptotische Schranke“ gefordert \\ $\impl \Theta(f(n))$ benötigt
		\pause
		\item Laufzeit eines Algorithmus soll angegeben bzw. bestimmt werden \\ \impl Offiziell $\Theta(f(n))$ erwünscht \\ (in VL oder Musterlösungen aber oft auch $O(f(n))$)
	\end{itemize}
\end{frame}

\sectionheadframe{Das Master-Theorem}{\thassedaniel{}{Bow, bow before your king, you shits!}}

\begin{frame}{Das Master-Theorem (einfache Form)} % TODO ceil the fraction?
	Seien $a, \textcolor{blue}{b}, c, \textcolor{darkgreen}{d}$ positive Konstanten und für $n \in \N$ sei 
	\[
	T(n) = 
	\begin{cases}
	a,  & \text{für } n = 1 \\
	\textcolor{darkgreen}{d} \cdot T\large(\frac{n}{\textcolor{blue}{b}}\large) + c\·n, & \text{für } n > 1
	\end{cases}
	\]
	gegeben. \\ \smallskip
	
	Dann gilt:
	\[
	T(n) \in 
	\begin{cases}
	\Th{n},                                                        & \textcolor{darkgreen}{d} < \textcolor{blue}{b} \\
	\Th{n \log n},                                                 & \textcolor{darkgreen}{d} = \textcolor{blue}{b} \\
	\Th{n^{\log _{\textcolor{blue}{b}} \textcolor{darkgreen}{d}}}, & \textcolor{darkgreen}{d} > \textcolor{blue}{b}
	\end{cases}.
	\]
\end{frame}

\begin{frame}[t]{Master-Theorem: Beispiele} 
	Wir betrachten verschiedene Sortierverfahren:\\
	\bigskip
	
	\begin{exampleblock}{Mergesort}
		\begin{algorithm}[H]
			\KwMethod Mergesort$(L: \|List| \: \KwSty{with} \size{L} = n)$\KwEmptyBlock{
				teile $L$ in der Mitte auf in $L_1$ und $L_2$ \;
				sortiere $L_1$ und $L_2$ rekursiv mit Mergesort \;
				füge $L_1$ und $L_2$ in $\Th{n}$ zusammen
			}
		\end{algorithm}
	\end{exampleblock}	
	
	\[\text{Laufzeit:} \quad T(n) = \pause \begin{cases}
	1, & n = 1\\
	2 \cdot T(\fract n/2 ) + 1 \cdot n, & n > 1
	\end{cases}\]
	
	\pause
	Nach MT (Fall 2) also: $T(n) \in \Th{n \log n}$.
\end{frame}

\begin{frame}[t]{Master-Theorem: Beispiele}
	Wir betrachten verschiedene Sortierverfahren:\\
	\bigskip
	
	\begin{exampleblock}{DoubleMergesort}
		\begin{algorithm}[H]
			\KwMethod DoubleMergesort$(L: \|List| \: \KwSty{with} \size{L} = n)$\KwEmptyBlock{
				teile $L$ in der Mitte auf in $L_1$ und $L_2$ \;
				sortiere $L_1$ und $L_2$ \textbf{jeweils zwei Mal} rekursiv mit DoubleMergesort \;
				\RComment{Ja, das ist natürlich konstruierter Blödsinn!} \;
				füge $L_1$ und $L_2$ in $\Th{n}$ zusammen
			}
		\end{algorithm}
	\end{exampleblock}	
	
	\[\text{Laufzeit:} \quad T(n) = \pause \begin{cases}
	1, & n = 1\\
	4 \cdot T(\fract n/2 ) + 1 \cdot n,  & n > 1
	\end{cases}\]
	
	\pause
	Nach MT (Fall 3) also: $T(n) \in \Th{n^{\log_2 4}} = \Th{n^2}$.
\end{frame}

\begin{frame}[t]{Master-Theorem: Beispiele}
	Wir betrachten verschiedene Sortierverfahren:\\
	\bigskip
		
	\begin{exampleblock}{Magicsort}
		\begin{algorithm}[H]
			\KwMethod Magicsort$(L: \|List| \: \KwSty{with} \size{L} = n)$\KwEmptyBlock{
				teile $L$ in der Mitte auf in $L_1$ und $L_2$ \;
				sortiere $L_1$ rekursiv mit Magicsort \;
				sortiere $L_2$ mithilfe eines Flaschengeistes (in \textbf{Nullkommanichts}!) \;
				füge $L_1$ und $L_2$ in $\Th{n}$ zusammen
			}
		\end{algorithm}
	\end{exampleblock}
	
	\[\text{Laufzeit:} \quad T(n) = \pause \begin{cases}
	1, & n = 1\\
	1 \cdot T(\fract n/2 ) + 1 \cdot n, & n > 1
	\end{cases}\]
	
	\pause
	Nach MT (Fall 1) also: $T(n) \in \Th{n}$.
\end{frame}

\iffalse  %TODO discard?

\begin{frame}{Beispiele Master-Theorem} 
	$n = 8^k, k \in \N_0$: \\[.5\baselineskip]
	$A(n) = $
	\begin{math}
		\begin{cases}
		42,                           & \text{für } n = 1 \\
		8 \cdot A(\frac{n}{8}) + 5n,  & \text{für } n > 1
		\end{cases}
	\end{math} \\[.5\baselineskip]
	\pause
	$\Impl A(n) \in \Theta(n\log n)$
\end{frame}

% add solutions to all


\begin{frame}{Beispiele Master-Theorem}
	$n = 4^k, k \in \N_0$: \\[.5\baselineskip]
	$B(n) = $
	\begin{math}
		\begin{cases}
		1337,                              & \text{für } n = 1 \\
		2 \cdot B(\frac{n}{4}) + 100000n,  & \text{für } n > 1
		\end{cases}
	\end{math} \\[.5\baselineskip]
	\pause
	$\Impl B(n) \in \Theta(n)$
\end{frame}


\begin{frame}{Beispiele Master-Theorem}
	$n = 2^k, k \in \N_0$: \\[0,25cm]
	$C(n) = $
	\begin{math}
		\begin{cases}
		69,                           & \text{für } n = 1 \\
		4 \cdot C(\frac{n}{2}) + 3n,  & \text{für } n > 1
		\end{cases}
	\end{math} \\[0,5cm]
	\pause
	$\Impl C(n) \in \Theta(n^{\log _{2}4}) = \Theta(n^2)$
\end{frame}


\begin{frame}{Beispiele Master-Theorem}
	$n = 13^k, k \in \mathbb{N}_0$: \\[0,25cm]
	$D(n) = $
	\begin{math}
		\begin{cases}
		8,                              & \text{für } n = 1 \\
		11 \cdot D(\frac{n}{13}) + 6n,  & \text{für } n > 1
		\end{cases}
	\end{math} \\[0,5cm]
	\pause
	$\Impl D(n) \in \Theta(n)$
\end{frame}


\begin{frame}{Beispiele Master-Theorem}
	$n = 3^k, k \in \mathbb{N}_0$: \\[0,25cm]
	$E(n) = $ 
	\begin{math}
		\begin{cases}
		255,                            & \text{für } n = 1 \\
		27 \cdot E(\frac{n}{3}) + 3n,   & \text{für } n > 1
		\end{cases}
	\end{math} \\[0,5cm]
	\pause
	$\Impl E(n) \in \Theta(n^{\log _{3}27}) = \Theta(n^3)$
\end{frame}


\begin{frame}{Beispiele Master-Theorem}
	$n = 35767^k, k \in \mathbb{N}_0$: \\[0,25cm]
	$F(n) = $
	\begin{math}
		\begin{cases}
		21,                                   & \text{für } n = 1 \\
		35767 \cdot F(\frac{n}{35767}) + 5n,  & \text{für } n > 1
		\end{cases}
	\end{math} \\[0,5cm]
	\pause
	$\Impl F(n) \in \Theta(n\log n)$
\end{frame}

\fi

\newcommand{\mastertheoreminder}{
	\scalebox{.8}{
		\begin{varwidth}{\columnwidth}
			Erinnerung Master-Theorem: \\
			$
			T(n) = \casesl{
				a,  & n = 1 \\
				\textcolor{darkgreen}{d} \cdot T\large(\frac{n}{\textcolor{blue}{b}}\large) + c\·n, & n > 1
			} 
			$ \hspace{-1.2\baselineskip} \\
			\impl 	$T(n) \in 
			\begin{cases}
			\Th{n},                                                        & \textcolor{darkgreen}{d} < \textcolor{blue}{b} \\
			\Th{n \log n},                                                 & \textcolor{darkgreen}{d} = \textcolor{blue}{b} \\
			\Th{n^{\log _{\textcolor{blue}{b}} \textcolor{darkgreen}{d}}}, & \textcolor{darkgreen}{d} > \textcolor{blue}{b}
			\end{cases}$. 	
		\end{varwidth}
	}
}

\begin{frame}{Aufgaben Master-Theorem}
	\taskheading{Master-Theorem}
	
	\begin{exampleblock}{Binäre Suche}
		\begin{algorithm}[H]
			\small
			\Function{$\|BinarySearch|(A: \KwSty{sorted} \KwArray[a..b] \KwOf \|Elem|, \, e: \|Elem|): \|Elem|$}{
				\begin{wrapfigure}{r}[0pt]{.4\textwidth}
					\vspace{-\baselineskip}
					\fbox{\mastertheoreminder}
				\end{wrapfigure}
				\eIf{$a = b$}{
					\Return{$\casesl{A[a], & \text{falls } A[a] = e \\ \bot, & \text{falls } A[a] \neq e}$}	
				}{
					$m := \floor{\fract a+b/2 }$ \; \smallskip
					\eIf{$e \leq A[m]$}{
						\Return{$\|BinarySearch|(A[a \dots m], \, e)$}
					}{
						\Return{$\|BinarySearch|(A[m+1 \dots b], \, e)$}
					}
				}
			}
		\end{algorithm}
	\end{exampleblock}
	Ermittelt die Laufzeit von BinarySearch.
\end{frame}

\begin{frame}[t]{Aufgaben Master-Theorem}
	\solutionheading \smallskip
	Laufzeit von BinarySearch ist $T(n) = \casesl{1, & n = 1 \\ 1 \· T(\fract n/2 ), & n > 1 }.$ \\
	\pause \smallskip
	\impl Master-Theorem sagt 
		\only<.(1)|handout:0>{$T(n) \in \Th{n}$}%
		\only<.(2)->{$\xcancel{T(n) \in \Th{n}}$}. \\
	\pause
	\impl \textbf{Reingefallen}! :P Master-Theorem will noch „$\vphantom{.} + c\·n$“. \textbf{Gibt's hier nicht}! \\ 
	\pause \medskip
	Hirn einschalten: BinarySearch \textbf{halbiert das Array} in jedem Durchlauf. \\
	Wie oft kann man $n := \size{A}$ halbieren? \\
	\pause \smallskip
	\impl $(\log_2 n)$-mal. \\
	\impl $T(n) \in \Th{\log n}$.
\end{frame}

\begin{frame}{Aufgaben Master-Theorem}  % TODO: maybe move to the beginning, if task sheet needs discussion
	\taskheading{Master-Theorem}
	\begin{wrapfigure}{r}[-.2\baselineskip]{.38\textwidth}
		\fbox{\mastertheoreminder}
	\end{wrapfigure}
	Die Laufzeit eines Algorithmus~A \\
	wird beschrieben durch \\ \smallskip
	\begin{math}
	U(n) = 
	\begin{cases}
	1,                           			   &n = 1 \\
	7 \cdot U(\ceil{\frac{n}{2}}) + n,  &  n > 1
	\end{cases}.
	\end{math} \\ \bigskip
	Ein weiterer Algorithmus~B hat die Laufzeit \\ \smallskip
	\begin{math}
	V(n) = 
	\begin{cases}
	1,                            				& n = 1 \\
	a \cdot V(\ceil{\frac{n}{4}}) + 5n,  & n > 1
	\end{cases}.
	\end{math} \\ \bigskip
	Was ist der größte Wert $a \in \N$, so dass B asymptotisch schneller als A ist?
\end{frame}

\begin{frame}{Aufgaben Master-Theorem}
	\solutionheading 
	\begin{itemize}
		\item Master-Theorem: Algorithmus A hat Laufzeit $\Theta(n^{\log _{2}7})$, wächst also stärker als $n^2$
		\pause
		\item Fall $a \leq 4$ also uninteressant $\impl a > 4$, d.~h. Algorithmus B läuft in $\Theta(n^{\log _{4}a})$ \\
		\pause 
		\item Also: 
		\begin{align*}
		&\log_{4}a < \log_{2}7 \gdw \frac{\log a}{\log 4} < \frac{\log 7}{\log 2} \gdw \log a < \log 7 \cdot \underbrace{\log 4}_{= 2}  \\ \pause
		\gdw &a < 2^{(\log 7) \cdot 2} = \left(2^{\log 7}\right)^{2} = 7^2 = 49 \Impl a = 48.
		\end{align*}
		(wobei \ $\log_x y = \frac{\log y}{\log x}$ \ und \ $\log x = \log_2 x$)
	\end{itemize}
\end{frame}


\begin{frame}{Aufgaben Master-Theorem}
	\taskheading{Master-Theorem}
	Gegeben sei folgende Rekurrenz für $n = 4^k, k \in \N_0$ \\[0,5cm]
	\begin{math}
	T(n) = 
	\begin{cases}
	2,                       & \text{für } n = 1 \\
	2 \cdot T(\frac{n}{4}),  & \text{für } n > 1
	\end{cases}
	\end{math} \\[0,5cm]
	Findet eine Funktion $f : \N \functionto \R^+$ und Konstanten $c_1, c_2$, so dass $c_1 \cdot f(n) \leq T(n) \leq c_2 \cdot f(n)$ und beweist euren Fund.
\end{frame}

\begin{frame}{Aufgaben Master-Theorem}
	\solutionheading \smallskip
	\hanging{\textbf{Behauptung} (Magie! \smiley): \quad $c_1 := 1$, $c_2 := 3$\ und\ $f(n) := \sqrt{n}$ 
	\newline erfüllen die Bedingung $\ \forall n = 4^k, k \in \N_0$} \\ \pause
	\textbf{Beweis} durch vollständige Induktion über $k$: \\ 
	\hanging{\textbf{IA}. $(k = 0 \impl n = 4^0 = 1$): \newline $1\cdot \sqrt{1} = 1 \leq T(1) = 2 \leq 3\cdot \sqrt{1} = 3$} \\ \pause
	\hanging{\textbf{IV}.: Für ein beliebiges, aber festes $k \in \N_0$\; $(n = 4^k)$ gelte \newline  $1 \cdot \sqrt{4^k} \leq T(n) \leq 3 \cdot \sqrt{4^k}$  \quad $\left(\gdw 1 \cdot \sqrt{n} \leq T(n) \leq 3 \cdot \sqrt{n}\right)$} \\ \pause
	\hanging{\textbf{IS}. ($k \rightsquigarrow k+1$): Es gilt: \vspace{-.5\baselineskip}
		\begin{align*}
			T\left(4^{k+1}\right) &\stackrel{Def.}{=} 2 \cdot T\left(\frac{4^{k+1}}{4}\right) = 2 \cdot T(n) \\ \pause
			2 \cdot T(n) &\stackrel{IV}{\geq} 2 \cdot 1 \cdot \sqrt{n} = \sqrt{4} \cdot \sqrt{n} = \sqrt{4n} = \sqrt{{4^{k+1}}} \\ \pause
			2 \cdot T(n) &\stackrel{IV}{\leq} 2 \cdot 3 \cdot \sqrt{n} = 3 \cdot \sqrt{4} \cdot \sqrt{n} = 3 \cdot \sqrt{4n} = 3 \cdot \sqrt{{4^{k+1}}}.\ \qed 
		\end{align*}
	}
\end{frame}
	
% TODO xkcd

\slideThanks
	
\end{document}