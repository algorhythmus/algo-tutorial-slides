%beamer

%\PassOptionsToClass{handout}{beamer}

\input{../preamble/tutpreamble}

% parent[] und d[] nachrüsten! S. Hinweis algotut10

\begin{document}
	
\starttut{9}
	
\sectionheadframe{Graphen durchlaufen}{Hänsel und Gretel im Tiefensuchwald}

\begin{frame}{Graphen durchlaufen}
	\begin{itemize}
		\item \textbf{Geg}.: Startknoten $s \in V$
		\item \textbf{Ziel}: Von $s$ aus alle weiteren Knoten besuchen
		\pause
		\item \textbf{Aber}: Keine \textbf{Doppel}besuche/Endlosschleifen \impl \textbf{Merke} besuchte Knoten
		\item Am Ende wollen wir \textbf{zu jedem Knoten nen Weg} haben
	\end{itemize}
\end{frame}

\morescalingdelimiters

\mycomment{  % Forest-Fassung:
	\Function{DFS-Forest$(G = (V, E)): (\KwArrayOf V, \KwArrayOf \N_0)$} {
		$visited = (\KwFalse, ... , \KwFalse): \KwArrayOf \|Boolean|$ \;
		$parent = (\bot, ... , \bot): \KwArrayOf V$, \qquad $d = (0, ..., 0): \KwArrayOf \N_0$  \;
		\smallskip
		\Procedure{DFS$(u,\, prev: V)$} {
			\If{$\neg visited[u]$} {
				visit$(u)$  \RComment{Do something with u} \;
				$visited[u] := \KwTrue, \quad parent[u] := prev, \quad  d[u] := d[prev] + 1$\;
				\ForEach{$(u, v) \in E$} {
					DFS$(v,\, prev = u)$
				}
			}
		}
		\smallskip
		\For{$v \in V$}{
			\If{$\neg visited[v]$}{
				DFS$(v,\, prev = v)$	
			}	
		}
		\Return{$(parent, d)$}
	}
}

\begin{frame}{Graphen durchlaufen – Tiefensuche}
	\begin{exampleblock}{Intuitive Implementierung: Tiefensuche\only<all:1>{, Beta-Version}\only<all:2>{, Release-Candidate}}
		\begin{algorithm}[H]
			\Function{DFS$(G = (V, E),\, s \in V)$}{
				$visited = (\KwFalse, ... , \KwFalse): \KwArrayOf \|Boolean|$ \;
				\visible<all:2>{$parent = (\bot, ... , \bot): \KwArrayOf V$, \qquad $d = (0, ..., 0): \KwArrayOf \N_0$}  \;
				\medskip
				\Procedure{DFS-step$(u\visible<all:2>{,\, prev} \in V)$} {
					\If{$\neg visited[u]$} {
						$visited[u] := \KwTrue\visible<all:2>{, \quad parent[u] := prev, \quad  d[u] := d[prev] + 1}$\;
						visit$(u\only<all:2>{,\, prev})$ \only<all:1>{\hphantom{$,\, prev$}} \RComment{Do something with u \only<all:2>{and prev}} \;
						\ForEach{$(u, v) \in E$} {
							DFS-step$(v\only<all:2>{,\, prev = u})$
						}
					}
				}
				\medskip
				DFS-step$(s\only<all:2>{,\,  prev = s})$ \;
				\visible<all:2>{\Return{$(parent, d)$}}
			}
		\end{algorithm}
	\end{exampleblock}
\end{frame}

\begin{frame}{Graphen durchlaufen – Tiefensuche}
	\taskheading{Tiefe in freier Wildbahn} 
	Führt auf diesem Graphen Tiefensuche von $s$ ausgehend aus und malt die Laufwege hinein. Nachbarn werden in alphabetischer Reihenfolge besucht.
	\vspace{-1.1\baselineskip}
	\begin{figure}[htp]
		\centering
		\includegraphics[height=6cm]{beispielgraph-edited}
	\end{figure}
\end{frame}

\begin{frame}{Graphen durchlaufen – Tiefensuche}
	\solutionheading
	Besuchsreihenfolge: \\ s, a, b, c, e, h, g, f, d, i, j, k
	\vspace{-.4\baselineskip}
	\begin{figure}[htp]
		\centering
		\includegraphics[height=6cm]{beispielDFS-edited}
	\end{figure}
\end{frame}

\begin{frame}[t]{Graphen durchlaufen – Tiefensuche}
	\begin{itemize}
		\item<+-> \textbf{Beobachtung}: Dringt schnell \textbf{tief} in den Graphen ein, anstatt sich „auszubreiten“ (daher der Name)
		\item<+-> \textbf{Laufzeit}? \visible<+->{$\Theta(n+m)$} \\ 
		\NoQuestionE{In-place?}{(wegen $visited$, $parent$ und Rekursion)}
		\item<+-> Etwas chaotische Laufwege -- geht's auch organisierter?
	\end{itemize}
\end{frame}

\begin{frame}{Graphen durchlaufen – Breitensuche}
	\begin{exampleblock}{Organisierte Reihenfolge: Breitensuche (einfach)}
		\begin{algorithm}[H]
			\Procedure{BFS$(G = (V, E),\ s \in V)$} {
				$visited := (\KwFalse, ... , \KwFalse)$\;
				$Q := \llist{s} : \|Queue|$ \;
				\While{$Q \neq \emptyset$} {
					$u := Q.\text{popFront}()$\;
					\If{$\neg visited[u]$} {
						visit$(u)$ \RComment{Do something with u} \;
						$visited[u] := \KwTrue$\;
						\ForEach{$(u, v) \in E$} {
							$Q.\text{pushBack}(v)$\;
						}
					}
				}
			}
		\end{algorithm}
	\end{exampleblock}
\end{frame}

\begin{frame}{Graphen durchlaufen – Breitensuche}
	\vspace{-1.3\baselineskip}
	\begin{exampleblock}{Organisierte Reihenfolge: Breitensuche \textbf{mit Buchhaltung}}
		\begin{algorithm}[H]
			\small 
			\Function{BFS$(G = (V, E),\ s \in V): (parent, d)$} {
				$visited := (\KwFalse, ... , \KwFalse), \quad \alert{parent := (\bot, ..., \bot), \quad d := (0, ..., 0)}$\;
				$Q := \llist{s}$ \;
				$\alert{parent[s] := s}$ \;
				\While{$Q \neq \emptyset$} {
					$u := Q.\text{popFront}()$\;
					\If{$\neg visited[u]$} {
						$visited[u] := \KwTrue, \quad \alert{d[u] := d[parent[u]] + 1}$ \;
						visit$(u, \alert{d[u]})$  \RComment{Do something with u and $d[u]$} \;
						\ForEach{$(u, v) \in E$} {
							\alert{$parent[v] := u$} \;
							$Q.\text{pushBack}(v)$  \;
						} 
					}
				} 
			}
		\end{algorithm} 
	\end{exampleblock}
\end{frame}


\begin{frame}{Graphen durchlaufen – Breitensuche}
	\vspace{-.3\baselineskip}
	\begin{exampleblock}{Organisierte Reihenfolge: Breitensuche \textbf{kompliziert} (siehe VL) }
		\begin{algorithm}[H]
			\small \vspace{-.4\baselineskip}
			\Function{BFS$(G = (V, E),\ s \in V): (parent, d)$} {
				$visited := (\KwFalse, ... , \KwFalse), \quad \alert{parent := (\bot, ..., \bot), \quad d := (0, ..., 0)}$\;
				$Q := \llist{s},$ \quad \alert{$Q' := \emptyset$}  \RComment{Extra queue Q'} \;
				$\alert{parent[s] := s, \quad layer := 0}$ \;
				\While{$Q \neq \emptyset$} {
					$u := Q.\text{popFront}()$\;
					\If{$\neg visited[u]$} {
						$visited[u] := \KwTrue, \quad \alert{d[u] := layer}$ \;
						visit$(u, \alert{layer})$  \RComment{Do something with u and layer} \;
						\ForEach{$(u, v) \in E$} {
							\alert{$parent[v] := u$} \;
							$Q'.\text{pushBack}(v)$  \RComment{Append to next-queue Q'} \;
						} \vspace{-.2\baselineskip}
					}
					\alert{
						\If{$Q = \emptyset$}{ 
							$(Q, Q') := (Q', Q)$ \RComment{New layer, so swap queues} \;
							$layer\pp$ 
						}
					} \vspace{-.2\baselineskip}
				} \vspace{-.2\baselineskip}
			}
		\end{algorithm} \vspace{-.5\baselineskip}
	\end{exampleblock}
\end{frame}

\begin{frame}{Graphen durchlaufen – Breitensuche}
	\taskheading{Volle Breitseite}
	Führt auf diesem Graphen Breitensuche von $s$ ausgehend aus. Nachbarn werden in alphabetischer Reihenfolge besucht.
	\vspace{-.3\baselineskip}
	\begin{figure}[htp]
		\centering
		\includegraphics[height=6cm]{beispielgraph-edited}
	\end{figure}
\end{frame}

\begin{frame}{Graphen durchlaufen – Breitensuche}
	\solutionheading
	Besuchsreihenfolge: \\ s,\quad a, c, d, \quad b, e, f, g, i, j, \quad h, k
	\vspace{-.3\baselineskip}
	\begin{figure}[htp]
		\centering
		\includegraphics[height=6cm]{beispielBFSwithLayers}
	\end{figure}
\end{frame}

\begin{frame}{Graphen durchlaufen – Breitensuche}
	\begin{itemize}
		\item<+-> \textbf{Beobachtung}: \textbf{Breitet} sich schnell stark \textbf{aus} (daher der Name)
		\item<+-> Offensichtlich: Findet \textbf{kürzeste Pfade} (bei \textbf{ungewichteten} Kanten)
		\item<+-> \textbf{Laufzeit}? \visible<+->{$\Theta(n+m)$} \\ 
		\NoQuestionE{In-place}{(wegen $visited$, $Q$ und $Q'$)}
	\end{itemize}
\end{frame}

\begin{frame}{Graphen durchlaufen – Generell}
	\begin{itemize}
		\item DFS/BFS finden \textbf{Pfade} von Startknoten $s$ zu allen anderen erreichbaren Knoten \\
		\impl $parent$-Array zum Rekonstruieren der Pfade \\
		{\small ($parent[v]$: Vorgänger von $v$ im Pfad zu $v$)}
		\item DFS/BFS messen „\textbf{Distanz}“ der Knoten \\
		\impl $d$-Array mit $d[v] = $ Anzahl Kanten auf dem Weg zu $v$ 
		\implitem \textbf{Rückgabewerte} von BFS/DFS im Pseudocode benutzbar: \\
		$(parent, d) := \|BFS|(G, s)$ \RComment{DFS similar} \\
		\LComment{Now use $parent[\cdot]$ and $d[\cdot]$} 
		\bigskip
		\pause
		\item DFS/BFS finden \textbf{nur} alle von $s$ \textbf{erreichbaren} Knoten
		\implitem Um den \textbf{ganzen Graphen} abzudecken, müsst ihr von jedem noch nicht erreichten Knoten \textbf{extra} loslaufen \: \emph{(„Tiefen-/Breitensuchwald“)}
	\end{itemize}
\end{frame}

\begin{frame}{Graphen durchlaufen – Kantentypen}
	\begin{itemize}
		\item Bei BFS/DFS „entlanggelaufene“ Kanten \textbf{bilden Baum} {\small (da kein Knoten zweimal besucht!)}
		\pause
		\implitem Teile Kanten ein:
	\end{itemize}
	\vspace{-\baselineskip}
	{\small \begin{description} 
			\setlength\itemsep{0pt} % Fuck U, LATEX!
			\setlength\topsep{0pt} 
			\setlength\parskip{0pt}
		\item[\textbf{tree}-:] „Entlanggelaufene“ Kanten des Baumes
		\pause
		\item[\textbf{cross}-:] Kanten \textbf{zwischen} versch. \textbf{„Ästen“} im Baum
		\pause
		\item[\textbf{backward}-:] Kanten, die \textbf{rückwärts} zu (einer/mehreren) \emph{tree}-Kanten laufen
		\pause
		\item[\textbf{forward}-:] Kanten, die \textbf{mehrere} \emph{tree}-Kanten „\textbf{überholen}“
	\end{description}}
	\vspace{-.5\baselineskip}
	\centering
	\visible<2->{\includegraphics[width=.6\textwidth]{edgetypes}}
\end{frame}

\begin{frame}{Graphen durchlaufen – Kantentypen}
	\taskheading{Die Graphschaft besichtigen} 
	Betrachtet die vorhin durchgespielte Tiefen- und Breitensuche und klassifiziert jeweils alle Kanten entsprechend. \vspace{-.2\baselineskip}
	\begin{figure}[htp]
		\centering
		\includegraphics[height=6cm]{beispielgraph-edited}
	\end{figure}
\end{frame}

\begin{frame}{Graphen durchlaufen – Kantentypen}
	\solutionheading
	Für Tiefensuche: \\
	\vspace{-.3\baselineskip}
	\begin{figure}[htp]
		\centering
		\includegraphics[height=6.5cm]{beispielDFStree-edited}
	\end{figure}
\end{frame}

\begin{frame}{Graphen durchlaufen – Kantentypen}
	\solutionheading
	Für Breitensuche:
	\vspace{-.3\baselineskip}
	\begin{figure}[htp]
		\centering
		\includegraphics[height=6.5cm]{beispielBFStree-edited}
	\end{figure}
\end{frame}

\begin{frame}[t]{Graphen durchlaufen – Kantentypen}
	\ContentQuestion{
		Gibt es eine Art von Kante, die bei Breitensuche nicht auftreten kann? Falls ja, warum?
	}{
		\textit{forward}-Kanten können \textbf{nicht} auftreten. \\
		(BFS bestimmt schon den Pfad mit kleinster Kantenanzahl.)
	}
	\ContentQuestion{
		Gibt es eine Art von Kante, die bei Tiefensuche nicht auftreten kann? Falls ja, warum?
	}{
		Bei Tiefensuche können \textbf{alle} Arten von Kanten auftreten.
	}
	
\end{frame}


\begin{frame}[t]{Graphen durchlaufen – Kantentypen}
	\ContentQuestion{
		Gibt es eine Art von Kante, die bei Tiefensuche \textbf{auf ungerichteten Graphen} nicht auftreten kann? Falls ja, warum?
	}{
		 \emph{cross}-Kanten können nicht auftreten: \\
		 Wäre nämlich schon \textbf{vorher} entlanggelaufen worden (da ungerichtet!). Die einzigen Kanten, die hier das Ende eines Tiefensuch-Astes markieren können, sind \emph{backward}-/\emph{forward}-Kanten. \\
		 (Ob man die jetzt \emph{backward}- oder \emph{forward}- nennt, ist wurscht, sind ja faktisch \textbf{beides}.) Bsp. dazu: 
		 \begin{center}
		 	 \includegraphics[width=.45\textwidth]{backward-dfs-undirected}
		 \end{center}
	}
\end{frame}

\begin{frame}[t]{Graphen durchlaufen – Kantentypen}
	\ContentQuestion{
		Sind \emph{cross}-Kanten eindeutig? Falls ja, warum?
	}{
		\emph{cross}-Kanten sind genau dann eindeutig, wenn der zugehörige Baum eindeutig ist. \impl I.~A. \textbf{nicht} der Fall (da Nachbarn i.~A. nicht in bestimmter Reihenfolge gewählt).
	}
	\ContentQuestion{
		Nach welcher Strategie muss bei Tiefensuche die Reihenfolge der rekursiven Abstiege (also die Reihenfolge der Nachbarn) gewählt werden, damit keine \textit{forward}-Kanten auftreten?
	}{
		\textbf{Fangfrage}! :P \\
		Es gibt \textbf{keine} solche Strategie; \emph{forward}-Kanten bei DFS in manchen Fällen unvermeidbar.
	}
\end{frame}


% Theoretisch prokrastinierbar, aber weils hier thematisch besser passt... Zeit langt vermutl. eh nicht:

\begin{frame}{Graphen durchlaufen}
	\taskheading{Tiefensuche revisited} 
	Implementiert Tiefensuche nicht-rekursiv als Pseudocode. Das asymptotische Laufzeitverhalten von eigentlicher Tiefensuche darf hierbei nicht überschritten werden.
\end{frame}

\begin{frame}{Graphen durchlaufen}
	\solutionheading
	Recursion-Faking mittels Stack: 
	\begin{algorithm}[H]
		\Procedure{DFS$\left(G=(V,E),\ s \in V\right)$}{
			$S := \llist{s} : \text{Stack}$ \;
			$visited := (\KwFalse, ..., \KwFalse)$ \;
			\While{$S \neq \emptyset$}{
				$u := S$.popBack$()$ \;
				\If{$\neg visited[u]$}{
					visit$(u)$ \RComment{Do something with u} \;
					$visited[u] := \KwTrue$ \; 
					\ForEach{$(u,v) \in E$}{
						$S$.pushBack$(v)$ \;
					}
				}
			}
		}
	\end{algorithm}
	\vphantom{\impl Tp}
\end{frame}

\begin{frame}{Graphen durchlaufen}
	\solutionheading
	Zum Vergleich: Breitensuche mit Queue
	\begin{algorithm}[H]
		\Procedure{BFS$(G = (V, E),\ s \in V)$} {
			$\alert{Q} := \llist{s}: \alert{\|Queue|}$ \;
			$visited := (\KwFalse, ... , \KwFalse)$\;
			\While{$\alert{Q} \neq \emptyset$} {
				$u := \alert{Q}.\text{pop\alert{Front}}()$\;
				\If{$\neg visited[u]$} {
					visit$(u)$ \RComment{Do something with u} \;
					$visited[u] := \KwTrue$\;
					\ForEach{$(u, v) \in E$} {
						$\alert{Q}.\text{pushBack}(v)$\;
					}
				}
			}
		}
	\end{algorithm}
	\impl Der Apfel fällt nicht weit vom Tiefensuchbaum... :P
\end{frame}

\xkcdframecustom{761}{}{2}{height=1.01\frameheight}

\only<beamer:0>{\slideThanks}

\end{document}