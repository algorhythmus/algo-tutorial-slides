%beamer

%\PassOptionsToClass{handout}{beamer}

\input{../preamble/tutpreamble}

\morescalingdelimiters

\begin{document}
	
\starttut{10}

\sectionheadframe{Kürzeste Pfade}{Es kommt halt \emph{doch} auf die Länge an...}

\section{Dijkstra}
	
\begin{frame}{Kürzeste Pfade – Dijkstra}
	\textbf{Der unaussprechliche Algorithmus} 
	\begin{itemize}
		\item Gesucht: \textbf{Kürzeste gewichtete} Pfade \\
		von Startknoten $s \in V$ zu \textbf{allen} anderen Knoten
		\pause
		\item \textit{Breitensuche}: Findet kürzeste Pfade bei \textbf{ungewichteten} Kanten
		\pause
		\implitem Passe BFS für gewichtete Kanten an: 
		\begin{itemize}
			\item $d[v]$: Länge des \textbf{bisher bekannten} kürzesten Pfades zu $v$ 
			\vspace{.2\baselineskip}
			\item $parent[v]$: \textbf{Direkter} Vorgänger von $v$ im \textbf{bisher bekannten} kürzesten Pfad zu $v$
		\end{itemize}
		\pause
		\item Rüste Queue $Q$ auf zu einer \textbf{PriorityQueue} $PQ$ (z.B. binärer Heap), Knoten $v$ wird mit $d[v]$ gewichtet
		%\pause
		%\item Invariante: Kürzester Pfad zu $PQ.min$ ist bekannt
		\pause
		\item \alert{Wichtige Einschränkung: \textbf{Keine negativen Kantengewichte!}}
	\end{itemize}
\end{frame}

\begin{frame}{Kürzeste Pfade – Dijkstra}  \vspace{-.25\baselineskip}
	\begin{exampleblock}{} \vspace{-.4\baselineskip}
		\begin{algorithm}[H]
			\small
			\Function {Dijkstra$(G = (V, E),\ s \in V)$} {
				$d := (\infty, ..., \infty) : \KwArray[1...n] \KwOf \R$\;
				$parent := (\bot, .., \bot) : \KwArray[1...n] \KwOf V$\;
				$PQ = \{s\} : $ \alert{PriorityQueue}\;
				$parent[s] := s, \quad d[s] := 0$ \;
				\While{$PQ \neq \emptyset$} {
					$u := PQ.\alert{\|deleteMin|()}$ \RComment{u wird jetzt „gescannt“} \;
					\ForEach(\RComment{„Relaxiere“ e}){$e = (u, v) \in E$} {
						\If{\alert{$d[u] + c(e) < d[v]$}} {
							$d[v] := \alert{d[u] + c(e)}$\;
							$parent[v] := u$\;
							\eIf{\alert{$v \in PQ$}} {
								\alert{$PQ.$decreaseKey$(v)$}\;
							}{
								$PQ.$insert$(v)$\;
							} \vspace{-.2\baselineskip}
						} \vspace{-.2\baselineskip}
					} \vspace{-.2\baselineskip}
				}
				\Return{$(d,\, parent)$}\;
			}
		\end{algorithm} \vspace{-.4\baselineskip}
	\end{exampleblock}
\end{frame}

\begin{frame}{Kürzeste Pfade – Dijkstra}
	\textbf{Korrektheit}
	\begin{itemize}
		%\pause
		\item \textbf{Invariante}: Wenn ein Knoten aus $PQ$ entnommen wird, ist zu diesem der \textbf{endgültige} kürzeste Pfad bekannt
		\pause
		\item Beweis der Invariante durch \textbf{vollständige Induktion} über die Schleifendurchläufe möglich
	\end{itemize}
\end{frame}

\iffalse

\mycomment{

	\begin{frame}{Kürzeste Pfade – Dijkstra}
		\textbf{Korrektheitsbeweis} \\
		%\begin{itemize}
		%\pause
		\underline{IA.:} Endgültiger kürzester Pfad zu $s$: Trivial \yop \\[0,125cm]
		\pause
		\underline{IV.:} Zu allen Knoten $v_1, ..., v_{i-1}$, die aus der $PQ$ entnommen wurden, \\
		\quad\ \ ist der \textbf{endgültige} kürzeste Pfad bekannt \\[0,125cm]
		\pause
		\underline{IS.:} Knoten $v_i$ wird entnommen. Der bekannte kürzeste Pfad führt \\
		\quad\ \ „irgendwie“ über $v_1 ... v_{i-1}$ zu $v_i$. \\
		\pause
		\quad\ \ Ang., es gibt einen \textbf{echt} kürzeren Pfad zu $v_i$. Dieser \textbf{muss} dann über \\
		\quad\ \ einen Knoten $p$ aus der $PQ$ zu $v_i$ führen. \\
		\pause
		\quad\ \ Dafür gibt es zwei Möglichkeiten: \\
		\pause
		\quad\ \ Fall 1: Zu $p$ gibt es einen kürzeren Pfad als zu $v_i$ \\
		\pause
		\qquad\qquad\ \ \impl $p$ wurde \textbf{vor} $v_i$ aus der $PQ$ entnommen \crash \\
		\pause
		\quad\ \ Fall 2: Der Pfad über $p$ zu $v_i$ ist kürzer als der kürzeste Pfad zu $p$ \\
		\pause
		\qquad\qquad\ \ \impl $c((p, v_i)) < 0$ \crash { \small(Keine negativen Kantengewichte erlaubt!)}
		%\end{itemize}
	\end{frame}
}

\fi

\begin{frame}[t]{Kürzeste Pfade -- Dijkstra}
	\textbf{Korrektheitsbeweis -- Invariante} \\ \smallskip
	\only<1-2|handout:1>{
		{\textbf{IA.:}} Endgültiger kürzester Pfad zu $s$: Trivial \yop \\
		\smallskip
		\pause
		{\textbf{IV.:}} Zu allen Knoten $v_1, ..., v_{i-1}$, die aus der $PQ$ entnommen wurden, \\
		\quad\ \ ist der \textbf{endgültige} kürzeste Pfad bekannt \\
	}
	\only<3-|handout:2->{
		{\textbf{IS.:}} Knoten $v_i$ wird entnommen. Der bekannte kürzeste Pfad zu $v_i$ führt \\
		\quad\ \ „irgendwie“ über $v_1 ... v_{i-1}$. \\
		\visible<4-|handout:2->{
				\quad\ \ Ang., es gibt einen \textbf{echt} kürzeren Pfad zu $v_i$. Dieser \textbf{muss} dann über \\
				\quad\ \ einen Knoten $p$ aus der $PQ$ zu $v_i$ führen. \\
				\visible<5-|handout:2->{
					\quad\ \ Dafür gibt es zwei Möglichkeiten: 
				} 
				\includegraphics[height=.47\frameheight]{dijkstra-correctness-figure0} \\
		} 
		
		\only<5|handout:2>{
			\quad\ \ \textbf{Fall 1}: Zu $p$ gibt es einen kürzeren Pfad als zu $v_i$ \\
			\qquad\quad\ \ \impl $p$ wurde \textbf{vor} $v_i$ aus der $PQ$ entnommen \crash \\
		}
		\only<6|handout:3>{
			\quad\ \ \textbf{Fall 2}: Der Pfad über $p$ zu $v_i$ ist kürzer als der kürzeste Pfad zu $p$ \\
			\qquad\quad\ \ \impl $c\tuple{(p, ..., v_i)} < 0$ \crash { \small(Keine negativen Kantengewichte erlaubt!)} \qed
		}
	}
\end{frame}

\begin{frame}{Kürzeste Pfade – Dijkstra}
	\textbf{Korrektheitsbeweis -- Terminierung} 
	\begin{itemize}
		\item In \textbf{jedem} Schleifendurchlauf wird ein Knoten $v$ aus $PQ$ entnommen. \\ \textbf{Endgültiger} kürzester Pfad zu $v$ bekannt \impl $v$ wird danach \textbf{nicht} mehr erneut in die $PQ$ eingefügt 
		\implitem Nach maximal $n$ Schleifendurchläufen ist $PQ$ leer und der Algorithmus terminiert. Hurra. \qed
	\end{itemize}
\end{frame}

\begin{frame}{Kürzeste Pfade – Dijkstra}
	\textbf{Laufzeit von Dijkstra} 
	\begin{itemize}
		\item[] Im Worst-Case $m$-mal $decreaseKey$
		\item[$+$] Genau $n$-mal $deleteMin$ und $insert$
		\pause
		\item[$=$] Mit binärem Heap: $O\left((m+n)\log n\right)$
		\pause
		\item[$=$] Mit Fibonacci-Heap: $O(m + n \log n)$ \quad (amortisiert und mit höheren konstanten Faktoren)
	\end{itemize}
\end{frame}

\begin{frame}{Kürzeste Pfade}
	\taskheading{Noch kürzere kürzeste Pfade} 
	Gegeben sei ein ( gerichteter oder ungerichteter) zusammenhängender Graph $G = (V, E)$ mit nichtnegativen Kantengewichten $\omega : E \functionto \R^{+}$. \\
	\smallskip
	 Beschreibt einen effizienten Algorithmus, der für einen Startknoten $s$ und alle Zielknoten $t \in V$ den Pfad mit den \textbf{wenigsten} Kanten unter allen kürzesten Pfaden von $s$ nach $t$ berechnet.
\end{frame}

\begin{frame}{Kürzeste Pfade}
	\solutionheading
	\textbf{Modifiziere} Dijkstra: Definiere Kantengewichte um als \textbf{Tupel} $c'(e) := \left(c(e), 1\right)$ (mit komponentenweiser Addition) und folgender Ordnung: \\ $(a, b) < (c, d) \Gdwdef a < c \ \lor\  (a = c \land b < d)$ 
\end{frame}



\section{Bellman-Ford}

\begin{frame}{Kürzeste Pfade – Bellman-Ford}
	\textbf{Rohe Gewalt: Bellman-Ford} 
	\begin{itemize}
		\item \textbf{Problem}: Dijkstra „erstickt“ an negativen Kantengewichten
		\pause
		\item \textbf{Überlegung}: Längster (zyklenfreier) Pfad hat \textbf{maximal} $n-1$ Kanten
		\item \textbf{Einleuchtend}: Jede Kante einmal zu relaxieren berechnet kürzeste Pfade der Länge 1 von jedem Knoten aus
		\pause
		\implitem Relaxiere jede Kante $(n-1)$-mal \impl \textbf{jeder} minimale zyklenfreie Pfad wurde bestimmt
		\pause
		\item \textbf{Laufzeit}: $O(n \cdot m)$
	\end{itemize}
\end{frame}

\begin{frame}{Kürzeste Pfade – Bellman-Ford}
	\begin{exampleblock}{}
		\begin{algorithm}[H]
			\small
			\Function {BellmanFord$(G = (V, E),\ s \in V)$} {
				$d := (\infty, ..., \infty) : \KwArray[1...n] \KwOf \R$\;
				$parent := (\bot, ..., \bot) : \KwArray[1...n] \KwOf V$\;
				$parent[s] := s ; \quad d[s] := 0$ \;
				\KwSty{do} $n-1$ \KwSty{times} \EmptyBlock{
					\ForEach{$e = (u, v) \in E$} {
						\If{$d[u] + c(e) < d[v]$} {
							$d[v] := d[u] + c(e)$\;
							$parent[v] := u$\;
						}
					}
				}
				\ForEach{$e = (u, v) \in E$} {
					\If{$d[u] + c(e) < d[v]$} {
						\LComment{kleinerer zyklenfreier Pfad ist nicht möglich \impl Negativer Zyklus!}
						$d[v] := -\infty$\;
					}
				}
				\KwRet{$(d,\, parent)$}\;
			}
		\end{algorithm}
	\end{exampleblock}
\end{frame}

\begin{frame}{Kürzeste Pfade}
	\taskheading{}
	Der ebenso geniale wie wirtschaftliche Superbösewicht Doktor Meta wittert das große Geld! Um seine Weltherrschaftspläne zu finanzieren, hat er einen teuflischen Plan ersonnen, mit dem er den internationalen Devisenhandel über den Tisch ziehen will: Zwischen je zwei Währungen $i$~und $j$ gibt es einen Wechselkurs, dargestellt als Umrechnungs-Faktor $T[i,j] > 0$. Doktor Meta will nun so lange Währungen tauschen, bis er am Ende mehr Geld in seiner Ausgangswährung „M\$“ hat als zuvor; das heißt, er sucht eine „Umtausch-Kette“, deren Gesamt-Faktor $> 1$ ist. \\
	\smallskip
	Sein größter Widersacher Turing-Man {\small (halb Mensch, halb Turingmaschine)} ist ihm jedoch auf den Fersen und will den Finanzmarkt gezielt aufmischen, um die Pläne des Superbösewichts zu vereiteln. Dafür braucht er allerdings die genaue Tauschreihenfolge von Währungen, die Doktor Meta ausnutzen will. Helft Turing-Man und  überlegt euch ein Verfahren, mit welchem ihr Metas genauen Plan enttarnen könnt.
\end{frame}

\begin{frame}{Kürzeste Pfade}
	\solutionheading
	\begin{itemize}
		\item Definiere $V := \set{\text{Währungen $i$}}$, $E := \set{(i, j) \Mid T[i,j] \neq \bot}$ und Kantengewichte $c(i, j) := -\log T[i,j]$. 
		\pause
		\item Starte \textbf{Bellman-Ford} von Knoten „M\$“ aus und suche nach einem \textbf{negativen Zyklus}. Dies ist der gewünschte Tauschzyklus.
		\pause
		\item [Denn:\!] Für eine Währungstauschfolge $(w_1, ..., w_\ell)$ mit $w_\ell = w_1$ gilt: 
		\begin{align*}
			& \prod_{i = w_1}^{w_{\ell-1}} T[i, i+1] \stackrel{!}{>} 1 \Gdw \log \prod_i T[i, i+1] > 0 \\ 
			{\Gdw} &-\log \prod_i T[i, i+1] < 0 \Gdw -\sum_i \log T[i, i+1] < 0 \\
			{\Gdw} &\sum_i c(i, i+1) < 0. \qed
		\end{align*}
	\end{itemize}
\end{frame}

\begin{frame}{Graphen}
	\taskheading{Travelling ferryman} % Dear Lord...
	Ein Fährmann soll einen Wolf, eine Ziege und einen Kohlkopf von der linken auf die rechte Seite eines Flusses befördern. Sein kleines Boot hat aber nur Platz für ihn und ein weiteres Objekt. \\
	Außerdem frisst der Wolf die Ziege und die Ziege den Kohlkopf, wenn der Fährmann nicht dabei ist. Zum Glück mag der Wolf kein Gemüse. Wie kann der Fährmann den Wolf, die Ziege und den Kohlkopf unbeschadet übersetzen? \\
	Löst das Problem mithilfe eines Graphen zeichnerisch.
\end{frame}

\begin{frame}{Graphen}
	\solutionheading
	Ein Zustandsgraph: \\
	\includegraphics[width=1.03\textwidth]{faehrmann} \\
	Weg von [WZKF|] nach [|WZKF] ist Lösung.
\end{frame}

\begin{frame}{Graphen}
	\taskheading{Domino-Day}
	Ihr habt folgende Dominosteine gegeben: \\
	\includegraphics[width=.7\textwidth]{domino-aufg} \\
	Ist es möglich, alle Steine als einen geschlossenen Ring anzuordnen, sodass nur gleiche Zahlen aneinanderliegen? Löst das Problem mithilfe eines Graphen zeichnerisch.
\end{frame}

\begin{frame}{Graphen}
	\solutionheading
	Pro Zahl ein Knoten, pro Stein eine Kante zwischen zwei Knoten. \\
	Zyklus $1 \rightarrow 4 \rightarrow 6 \rightarrow 2 \rightarrow 5 \rightarrow 3 \rightarrow 1 \rightarrow 2 \rightarrow 4 \rightarrow 5 \rightarrow 1$ ist Lösung. \\
	\includegraphics[width=.4\textwidth]{domino-lsg} \\
	\pause
	Allgemein lösbar, wenn Graph zusammenhängend ist und eulerschen Kreis enthält (\gdw nur gerade Knotengrade enthält). \\
	{\small (siehe Exkurs nächste Folie)}
\end{frame}

\begin{frame}{Exkurs Eulerkreis/Hamiltonkreis}
	\textbf{Eulerkreis}: \\
	Kreis, der jede \textbf{Kante} genau einmal beschreitet. {\small (\textbf{E}uler \impl \textbf{E}dges \impl Kanten)} \\
	\forcenewline
	\pause
	\textbf{Hamiltonkreis}: \\
	Kreis, der jeden \textbf{Knoten} genau einmal beschreitet. {\small (\textbf{H}amilton \impl \textbf{H}noten :P)} \\
	\forcenewline
	\pause
	$\exists$ Eulerkreis in $G$ \gdw $G$ hat nur gerade Knotengrade. \\
	$\exists$ Hamiltonkreis in $G$ \gdw Ausprobieren! :P (Gibt kein einfaches Kriterium)
\end{frame}

\xkcdframevert{612}{Danke für eure Aufmerksamkeit! \smiley}{2}

\only<beamer:0>{\slideThanks}

\end{document}